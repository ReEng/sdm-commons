\chapter{Related Work} \label{sec:RelatedWork}

\todojr{Write overview of the chapter.}

\section{Origins and Previous Work on Story Diagrams (Jan)}

Story diagrams have been described first by Fischer et al. \cite{FNTZ00} and Jahnke and Z\"{u}ndorf \cite{JZ98} in 1998.
The foundations of story diagrams lie in the programmed graph rewriting systems PROGRES \cite{SWZ95} which has been developed at the University of Aachen since 1989.
Story diagrams (or story flow diagrams, as they have been called in early publications) adapt and enhance the PROGRES approach to a UML-like notation and an object-oriented data model \cite{JZ98}.
They have an easily comprehensible graphical syntax and well-defined semantics.
Z\"{u}ndorf \cite{Zun01} describes the syntax and semantics of story diagrams in detail.
A graph grammar that formally describes the syntax of the control flow of story diagrams was defined by Klein \cite{Kle99}.

Story diagrams are embedded in a rigorous and systematic software development method called \emph{story-driven modeling} (SDM) \cite{Zun01,DGZ04}.
While existing approaches like UML focus on the specification of the static structure of software, SDM combines, amongst others, UML class diagrams, UML activity diagrams, and story diagrams to allow completely specifying the structure and behavior of software systems.
Furthermore, SDM describes how such a software specification can be derived from requirements.
First, each use-case in the requirements is refined by a set of sample scenarios defined by so-called \emph{story boards}.
A story board is a sequence of single snap shots of graph-like object structures, describing changes in these object structures.
Next, the static class structure of the system is derived from the story boards and further refined.
Given the sample scenarios, the general dynamic behavior of the system is then defined using story diagrams.
Finally, the implementation of the software system can be automatically generated from these formal models.

From the beginning, tool support for story diagrams was a main focus.
Fujaba, an acronym for ``From UML to Java And Back Again''\footnote{The acronym is derived from a preceding tool called FUCABA (''From UML to C++ And Back Again'') \cite{JZ97}.}, was the first tool which implemented the concept of story diagrams.
In December 1997, the project started at the University of Paderborn.
A first prototype was implemented in the course of a master thesis \cite{FNT98}.
As story diagrams specify the behavior of software, the execution of story diagrams is an important requirement.
For instance, Z\"{u}ndorf, Sch\"{u}rr and Winter \cite{ZSW99} describe how story diagrams can be compiled into Java code.
This code generation approach was also integrated into Fujaba.

A first public tool demonstration of Fujaba was presented at the ICSE 2000 \cite{NNZ00}, showing advanced class and story diagram modeling facilities as well as graphical debugging and simulation.

In the following, story diagrams and Fujaba have been modified and enhanced.
Originally, story diagrams use expressions of the target programming language to define constraints, return values etc.
I.e., if a story diagram should be compiled into Java code, Java expressions must be used.
St\"{o}lzel, Zschaler and Geiger \cite{SZG07} integrated OCL into story diagrams, making them more platform-independent.
They connected Fujaba with the Dresden OCL toolkit \cite{DresdenOCL}, allowing a code generation for story diagrams including the OCL constraints.

To improve flexibility for the execution of story diagrams, Giese, Hildebrand and Seibel \cite{GHS09} present an interpreter for story diagrams.
In contrast to executing generated Java code, with this approach generated story diagrams can be executed immediately.
This allows, for instance, to create higher-order transformations where story diagrams are created by other story diagrams and immediately be executed.
As interpreting in general is slower than compiling, the authors implemented a new dynamic matching policy for their interpreter.

Tichy, Meyer and Giese \cite{TMG06} identified some semantic issues in story diagrams.
First, when creating more than one element in a story pattern, the order of creation is undefined.
In general, this is no problem; however, in certain failure situations and when creating links in ordered associations, this may lead to non-deterministic behavior.
We deal with this issue as described in \todojr{Section X}.
Second, when having a link between two set variables \fe{setA} and \fe{setB}, the intuitive semantics would be to have every set element in \fe{setA} connected to every element in  \fe{setB}.
However, this is neither supported by the tools nor allowed by the formal semantics described by Z\"{u}ndorf \cite{Zun01}.
\todojr{describe solution}
Third, consider there is a class with two qualified associations (to other classes) that have each other as a qualifier.
When creating one link for each of the two qualified association in one story pattern, the first association that is created is qualified by the \fe{null} value although it could be qualified using the correct object (considering this is already bound).
We discuss this issue in \todojr{Section X}.
Forth, the set of possible bindings that match in a for-each activity may be extended by this very for-each activity, i.e., the activity changes something that makes new elements match for the for-each condition.
Thus, we clarify whether we use a \emph{pre-select} or a \emph{fresh matches} semantics for for-each activities in \todojr{Section X}.
Fifth, as creations may fail, e.g., due to resource constraints, the authors propose that a story diagram should be able to react to the result of a creation.
We mention this in \todojr{Section X}.

The control flow of story diagrams is modeled explicitly.
However, in certain situations, it is useful to only implicitly define the execution order, as it may significantly improve the comprehensibility of a story diagram.
Thus, Meyers and Van Gorp \cite{MG08} propose to add a new language construct for the non-deterministic selection of a execution order.

In~\cite{Sta08}, Stallmann presents an extension of story patterns which is called \emph{enhanced story pattern}. They extend story patterns by so-called \emph{insets}. Insets carry a qualifier which applies to all object and link variables in the inset. That allows to mark sub-graphs as negative, to specify \emph{and} and \emph{or} conditions on subgraphs and to qualify a subgraph by $\forall$. We will adopt these ideas in future versions of this document.

Becker et al. present means for structuring complex transformations into several independent story diagrams which can be called in a well-defined manner \cite{BvDHR11}.
They propose inventing explicit call activities which invoke other story diagrams and also support polymorphic dispatching.
Polymorphic dispatching can also be used for the aforementioned case of non-deterministic execution order.
We adopt this proposal in \todojr{Section X}.

Until 2010, different branches of story diagrams and of Fujaba were developed, leading to severe difficulties when exchanging data due to incompatibilities.
In an effort to again unify the different branches, a task force was started in 2010.
A first result of this joint effort of the SDM community was a new unified and consolidated meta-model for story diagrams based on EMF \cite{HRvD+11}.
This new meta-model is the foundation for future projects; this technical report is also based on this meta-model.
One extension is the support for explicitly modeling expressions.
However, this is not described in detail here.



%motivation for the acronym Fujaba, \textit{F}rom \textit{U}ML to \textit{J}ava \textit{a}nd \textit{b}ack \textit{a}gain \cite{JZ97}

%story-driven modelling and story boards (roots of Fujaba) \cite{FNT98,FNTZ00,JZ98,ZSW99,DGZ04}

%story diagrams \cite{FNTZ00,Zun01}

%SD graph grammar for the construction of valid SDs (besides others) \cite{Kle99}

%Fujaba tool demo \cite{NNZ00}


%semantic issues \cite{TMG06}

%new meta-model \cite{HRvD+11}



\section{Applications and Extensions of Story Diagrams (Jan)}
Giese and Klein extend story patterns to so called Story Decision Diagrams (SDDs) that allow to express complex safety properties \cite{GK06a}.
Basically, they require story patterns of SDDs to be non-modifying (i.e., no \create or \destroy elements) and add features of logics such as quantification, implication, and negation.
After the evaluation of such a property, a regular story pattern may be specified which describes a change operation that should be executed.

Giese and Klein also present Timed Story Scenario Diagrams (TSSDs), which are used to specify structural and temporal properties of systems in an integrated way \cite{KG07a}.

Tichy et al. \cite{THH+08} describe how story diagrams can be used to describe reconfigurations of component-based architectures, as, for instance, in MechatronicUML \cite{BBD+12}.
A transformation language called Component Story Diagrams is used to specify reconfiguration steps.
Component Story Diagrams use the concrete syntax of components for specifying the reconfiguration operations.
This language is transformed to story diagrams that can be executed to perform the actual reconfigurations.

Z{\"u}ndorf~\cite{Zue09} proposed a framework for computing the state-space of a specification in terms of story diagrams and an initial instance model. 
It has been used for model checking the leader election protocol and for a case study presented in~\cite{HSJZ10}.

In~\cite{HH11b}, Heinzemann and Henkler extend story diagrams to timed story diagrams. 
Timed story diagrams are based on timed graph transformation systems~\cite{EHH+11} that extend graph transformation systems by clocks as known from timed automata~\cite{AD94}. 
They are used to model time-dependent reconfigurations of an instance model. 
In~\cite{EHH+11}, they are used as a means to define the semantics of reconfigurations in real-time systems. 
In~\cite{HSE10}, a framework for reachability analysis on timed story diagrams has been introduced which explores the state-space defined by the timed story diagrams and an initial instance model. 
It is based on the framework introduced in~\cite{Zue09}.

\begin{itemize}
\item Design-level debugging \cite{GZ02,Gei02,GZ06},
\item code generation \cite{GSR05,GBD07},
\item reverse engineering \cite{NSW+02,BGS+Z08}, MATE (reverse engineering) \cite{SKS+07,ST08},
\item OCL in Fujaba \cite{SZG07},
\item other applications \cite{KNNZ00,GZ10}
\end{itemize}


\section{Work Related to Story Diagrams (Jan)}

Model transformation has become an important research topic during the last years.
Several concepts and tools with different scopes and applications have been proposed.

Several model transformation approaches exist which are similar to story diagrams.

Here, we focus on those solutions that have a reasonable documentation available.
For a more comprehensive overview of transformation approaches see, for example, \cite{Czarnecki06}.
Current transformation tools can, for instance, be found in \cite{TTC2010}.

\subsection{Endogenous, In-Place Model Transformations}

\emph{Henshin}~\cite{henshin2} is a model transformation language for in-place transformations of EMF-based models.
It uses pattern-based rewrite rules (called ``transformation rules'') and control-flow-based operational semantics (called ``transformation units'') on top of it.
Transformation units can also be called by other transformation units, also including parameters.
%Henshin, however, does not provide support for polymorphic dispatching.

\emph{MOLA}~\cite{mola} is an in-place model transformation language with graphical syntax similar to story diagrams.
Transformation rules may consist of multiple matching and modification patterns, and the control flow inside a transformation rule can be specified, with a focus on the loop construct.
Furthermore, it also allows calling other transformations rules. %, but does not support polymorphic dispatching.

\emph{Groove}~\cite{Ren04a} is a graph transformation tool with a focus on analyzing graph transformation systems.
Its rules consist of single rewrite patterns.
For instance, given a rule set and a start graph, Groove can explore the graph state space and use this for model checking.
It also features so called ``control programs'', which allow the user to restrict which rules can be applied and in which order. In provide model checking of LTL properties~\cite{Ren08} and CTL properties~\cite{KR06}.

\emph{VIATRA}~\cite{viatra}, a textual language, uses abstract state machines to specify the control flow and graph transformation rules for elementary model manipulations.
It also addresses modularization by reusable patterns that are called from the graph transformation rules. 

%Although most of the story-diagram-like transformation languages include means for specifying control flow (including calling other transformation units), none of them supports polymorphic dispatching.
%(Note that in most cases polymorphic dispatching can be emulated using other means, but doing so would result in a more complex and less maintainable rule set.)

%However, looking at other transformation language types, there are some approaches that support polymorphic dispatching. 
%For instance, \emph{Xpand}~\cite{xpand}, a model-to-text transformation language based on templates, uses ``polymorphic template invocation'' where the most %specialized template available is used.
%However, it only supports single dispatch, i.e., only one parameter is used to determine the used template.   

\subsection{Exogenous, Inter-Model Transformations}

In general, Story Diagrams can also be used to specify inter-model transformations.
In this case, a story diagram would contain elements from both the source and the target model.
If necessary, a trace model could also be created.
In comparison to dedicated inter-model transformation languages, story diagrams may be more tedious in this application scenario.
However, when a transformation requires extensive pre-computations or complex distinction of cases, story diagrams are a reasonable alternative.

\emph{QVT Operational}~\cite{QVT} is a operational model transformation language designed for writing unidirectional transformations which is part of the OMG QVT standard. %that also allows polymorphic dispatching by its \fe{disjuncts} keyword.
%A QVT-O mapping operation can declare that a call to it should be dispatched to other mappings.
%In this case, the invocation of that mapping operation results in the execution of the first mapping whose signature fits the concrete parameters and whose \fe{when} clause evaluates to \fe{true}.
%This is a more powerful construct than our solution, as it not only allows dispatching based on the actual type, but arbitrary constraints.
%However, there must be a base rule where all dispatching possibilities are listed;
%in our solution, all signature-compatible transformations with the same name are automatically used in dispatching, allowing a better modularization as well as an easier extension of the rule set.      
However, QVT-O is a textual transformation language which may not be well-suited in many cases~\cite{Moo09}.

In declarative inter-model transformation languages like \emph{Triple Graph Grammars} (TGGs) \cite{Sch94}, the control flow cannot be defined explicitly.
Instead, the order of the rule application is implicitly defined by preconditions of the transformation rules.
However, when more than one rule has a fitting precondition, the rule to be applied is selected non-deterministically, dependent on the concrete transformation tool implementation, or by a given rule priority.
This can make the comprehension of a TGG rule set difficult.
%Klar et al. \cite{Klar07} proposed a rule generalization concept with a precedence for the most refined rules, a solution similar to polymorphic dispatching.

In \emph{QVT Relations}~\cite{QVT}, which is similar to TGGs, control flow may also be explicitly specified by using \fe{where} clauses.

The \emph{Atlas Transformation Language} (ATL)~\cite{ATL} is a hybrid inter-model transformation language, integrating declarative and operational aspects.
It is similar to QVT, but only has a textual representation of the transformation rules.

\section*{Further related work}

\begin{itemize}
\item AGG
\item Evtl. Loesungen zu M2M Transformation Aufgabe aus dem Tool Transformation Contest 2010 fuer weiteres Related Work anschauen
\item GReTL \cite{HE11}
\end{itemize}