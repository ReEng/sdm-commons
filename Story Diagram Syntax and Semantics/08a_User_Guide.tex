\chapter{User Guide}\label{sec:UserGuide}
	
This section should contain a user guide for the available story diagram tools.


\section{Installation}


\subsection{Installation Using the Eclipse Update Site -- Users}


\subsection{Getting the Source Code From Repository -- Developers}


\section{Getting Started -- User Interface}

\subsection{Story Diagram Editor}

\subsection{Story Diagram Interpreter (Stephan)}

The story diagram interpreter is split into a core and a metamodel-specific connector. For more details, see Sect.~\ref{sec:InterpretingStoryDiagrams}. In addition, the interpreter can be used inside Eclipse, which is the easiest way, or stand-alone without Eclipse, which requires some additional steps.

\subsubsection{Using the interpreter in Eclipse}

To use the interpreter in Eclipse in your own code, simply create a new instance of \emph{StoryDrivenEclipseInterpreter} and execute its \emph{executeActivity} operation. The \emph{ClassLoader} parameter is required to allow executing Java operations via reflection in Story Diagrams.

\begin{verbatim}
//Get the activity to execute
Activity activity = ...;

//Compile the list of parameters of the activity
List<Variable<EClassifier>> parameters = ...;

//Create the interpreter
StoryDrivenEclipseInterpreter interpreter = 
  new StoryDrivenEclipseInterpreter(getClass().getClassLoader());
	
//Execute the activity
interpreter.executeActivity(activity, parameters);
\end{verbatim}

The interpreter also provides a notification mechanism. Clients can register a \emph{NotificationReceiver} at the interpreter to be informed of all
relevant execution steps. This interface defines a \emph{notifyChanged} method that receives all kinds of \emph{InterpreterNotifications}.
\emph{StoryDrivenOutputStreamNotificationReceiver} is an example implementation of \emph{NotificationReceiver} that writes all notifications
to an output stream (or \emph{System.out} if none specified.).

\begin{verbatim}
//Create the interpreter like above
StoryDrivenEclipseInterpreter interpreter = 
  new StoryDrivenEclipseInterpreter(getClass().getClassLoader());

//Register a NotificationReceiver
interpreter.getNotificationEmitter().addNotificationReceiver(
  new StoryDrivenOutputStreamNotificationReceiver());
\end{verbatim}

\subsubsection{Using the interpreter without Eclipse}

If the interpreter is used without Eclipse, two things have to be noted. 
First, while the SDM metamodel is based on EMF, the initialization normally done by EMF automatically, has to be performed manually. 
This is done by accessing the metamodel package's \emph{eINSTANCE} field and registering the appropriate resource factory. 
This is also required for all other EMF-based metamodels that are used in the application.

\begin{verbatim}
//Create a ResourceSet.
ResourceSet resourceSet = new ResourceSetImpl();

//Register the XMIResourceFactory as the default resource 
//factory for *.sdm files.
resourceSet.getResourceFactoryRegistry().
  getExtensionsToFactoryMap.put("sdm", 
  new XMIResourceFactoryImpl());

//Register the org.storydriven.core.CorePackage package 
//with its nsURI in the package registry.
resourceSet.getPackageRegistry().
  put(CorePackage.eNS_URI, CorePackage.eINSTANCE);

//Register all the meta model packages that you will 
//use with their nsURIs in the package registry.
resourceSet.getPackageRegistry().
  put(YourPackage.eNS_URI, YourPackage.eINSTANCE);
\end{verbatim}

Second, all expression interpreters responsible for evaluating expressions in a story diagram (e.g., OCL expressions) have to be registered manually with the SDM interpreter.

\begin{verbatim}
//Create the expression interpreter manager
StoryDrivenExpressionInterpreterManager 
  expressionInterpreterManager = 
  new StoryDrivenExpressionInterpreterManager(
  getClass().getClassLoader());
	
//Register expression interpreters for specific expression 
//languages
expressionInterpreterManager.registerExpressionInterpreter(
  new OCLExpressionInterpreter<Expression>(), "OCL", "1.0");

//Create the interpreter and execute the story diagram
StoryDrivenInterpreter interpreter = 
  new StoryDrivenInterpreter(expressionInterpreterManager);
\end{verbatim}

