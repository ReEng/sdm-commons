\documentclass[12pt,a4paper,twoside,titlepage,openright,headsepline,listof=totoc,index=totoc,chapterprefix,bibliography=totoc]{scrreprt}


\usepackage[ngerman,english]{babel}
%\selectlanguage{english}
\usepackage[T1]{fontenc}


\newif\ifpdfhere
\ifx\pdfoutput\undefined
\pdfherefalse % we are not running pdflatex
\else
\pdfoutput=1 % we are running pdflatex
\pdfcompresslevel=9     % compression level for text and image;
\pdfminorversion=6
\pdfheretrue
\fi

\ifpdfhere
%\usepackage{techreport}

\usepackage[pdftex,bookmarks=true,bookmarksopen=false,bookmarksnumbered=true,linktocpage,colorlinks=true,backref,pagebackref, linkcolor=blue,  citecolor=blue, urlcolor=blue]{hyperref}
%% Weitere Hyperrefeinstellungen
\hypersetup{%
pdfauthor={Markus von Detten, Stephan Hildebrandt, Christian Heinzemann, Marie Christin Platenius, Jan Rieke, Julian Suck, Dietrich Travkin}
,pdftitle={Story Diagrams -- Syntax and Semantics}
,pdfsubject={Technical Report}
%% fuer die Screen-Version: blue
,linkcolor=blue,anchorcolor=blue,citecolor=blue,filecolor=blue,menucolor=blue,urlcolor=blue
% fuer die Print-Version: black
% ,linkcolor=black,anchorcolor=black,citecolor=black,filecolor=black,menucolor=black,pagecolor=black,urlcolor=black%
}

\usepackage[pdftex]{graphicx}
\usepackage{thumbpdf}

\graphicspath{{figures/}}

\fi


% *** amsfonts ***
% Paket  : amsfonts
% Zweck  : f"ur spezielle Mathematik-Schriftarten (z.B. geschwungenes NP)
% Hinweis: Die Symbole erh"alt man mit \mathcal{NP}
% Doku   : Dokumentation zu finden unter (SuSE-Linux 7.0):
%          /usr/share/doc/packages/te_latex/texmf/fonts/amsfonts/amsfndoc.dvi
\usepackage{amsfonts}
\usepackage{algorithmic}
\usepackage{listings}
\lstset{ % 
basicstyle=\footnotesize,
tabsize=3,
language=[Visual]C++,
backgroundcolor=\color{white}
}
\usepackage{url}
\usepackage{color}
\usepackage{tabularx}
\usepackage[normalem]{ulem}
\usepackage{inputenc}

\usepackage{std_authorlist}

% *** amsmath ***
% Paket  : amsmath
% Zweck  : f"ur spezielle Mathematik-Symbole (z.B. nat"urliche und reelle Zahlen)
% Hinweis: Die Symbole erh"alt man mit \mathbb{N}
% Doku   : Dokumentation zu finden unter (SuSE-Linux 7.0):
%          /usr/share/doc/packages/te_latex/texmf/latex/amsmath/amsldoc.dvi
\usepackage{bbm}

\usepackage{amsmath}
\usepackage{amssymb}
\usepackage{ntheorem}
\usepackage{mathtools}

% *** fancyvrb ***
% Paket  : fancyvrb
% Zweck  : f"ur das Einf"ugen von Quellcode
% Doku   : Dokumentation zu finden unter (SuSE-Linux 7.0):
%          /usr/share/doc/packages/te_latex/texmf/latex/fancyvrb/fancyvrb.ps
\usepackage{fancyvrb}

% *** floatflt ***
% Paket  : floatflt
% Zweck  : f"ur Textfluss um Abbildungen und Tabellen
% Doku   : Dokumentation zu finden unter (SuSE-Linux 7.0):
%          /usr/share/doc/packages/te_latex/texmf/latex/floatflt/floatflt.dvi
\usepackage{floatflt} 

% *** graphics ***
% Paket  : graphicx
% Zweck  : zum Einbinden von EPS-Graphiken
% Doku   : Dokumentation zu finden unter (SuSE-Linux 7.0):
%          /usr/share/doc/packages/te_latex/texmf/latex/graphics/grfguide.ps
\usepackage{graphicx}

% \DeclareGraphicsExtensions{.png,.jpg}
% *** ngerman ***
% Paket  : ngerman
% Zweck  : f"ur Umlaute, deutsche Silbentrennung usw.
% Hinweis: ngerman benutzt im Gegensatz zu german die neue Rechtschreibung
%          (wichtig f"ur die Silbentrennung)
% Doku   : Dokumentation zu finden unter (SuSE-Linux 7.0):
%          /usr/share/doc/packages/te_latex/texmf/generic/styles/gerdoc.dvi
%\usepackage{ngerman}
%\selectlanguage{english} % Dokumentensprache auf deutsch stellen


% *** times ***
% Paket: Times
% Zweck: Schrift Times benutzen (sieht wesentlich besser als die Standardschrift aus)
\usepackage{times}
\usepackage[rightcaption]{sidecap}

%\usepackage{setspace}

\addtolength{\oddsidemargin}{1.5cm}
\addtolength{\evensidemargin}{-1.5cm}
%\addtolength{\topmargin}{-0,2cm}
%\addtolength{\textheight}{+1.5cm}
%\addtolength{\textwidth}{-0cm}

\sloppy
\usepackage{supertabular} 

%% snake-oil f�r den Satz
 \pretolerance=100           %% Textsatz: interner Parameter zur Steuerung des Zeilenumbruchs \tolerance 300              %% 1414 Bewertungsgrenzwert f�r schlecht umbrochene Zeilen
 \hfuzz=0.2pt                %% Grenze, ab der eine overfull hbox gemeldet wird
 \vfuzz=0.2pt                %% Grenzwert, ab dem die �berf�llung einer \vbox protokolliert wird
 \hbadness 1414              %% Grenzwert f�r �schlechte� Zeilen, bzw. Boxen
 \vbadness	1000              %% Grenzwert f�r eine �schlechte� \vbox 
 \emergencystretch 1em       %% zus�tzlicher dynamischer Leerraum
 \hyphenpenalty=30           %% Strafpunkte bei Silbentrennung �ber Absatz hinweg
 \widowpenalty=100000          %% falls letzte Zeile auf neue Seite gebrochen wird.
 \clubpenalty=100000           %% wenn erste Zeile eines Absatzes auf alter Seite bleibt.
 \doublehyphendemerits=50    %% Aufeinanderfolgende Silbentrennungen eher vermeiden. 

\setcounter{secnumdepth}{3}

%use this command to denote figure element that appear inside a figure
\newcommand{\fe}[1]{\textsf{\small #1}}

\theoremstyle{break}
\newtheorem{definition}{Definition}[section] 
\newtheorem{bsp}{Beispiel}[section] 

\usepackage{fancyhdr}

\pagestyle{fancy} 
\fancyhf{} 
\fancyhead[EL,OR]{\thepage} 
\fancyhead[ER]{\small\scshape{\leftmark}}
\fancyhead[OL]{\small\scshape{\rightmark}}

%% \TODO{}-Command um schnell zu sehen was noch zu machen ist ;)

\RequirePackage{xspace}
\providecommand{\fuj}[1][\xspace]{{\scshape FUJABA}#1}
\providecommand{\destroy}[1][\xspace]{{\texttt{\small{\flqq destroy\frqq}}}#1}
\providecommand{\create}[1][\xspace]{\texttt{\small{\flqq create\frqq}}#1}

\setlength{\marginparwidth}{2cm}

%\usepackage[colorinlistoftodos, textwidth=2.0cm, disable]{todonotes}
\usepackage[colorinlistoftodos, textwidth=2.0cm]{todonotes}
\newcommand{\todoall}[1]{\todo[inline, color=green!40]{Alle: #1}}
\newcommand{\todoch}[1]{\todo[inline, color=yellow!40]{Chris: #1}}
\newcommand{\todomvd}[1]{\todo[inline, color=cyan!40]{Markus: #1}}
\newcommand{\tododt}[1]{\todo[inline, color=blue!40]{Dietrich: #1}}
\newcommand{\todoib}[1]{\todo[inline, color=pink!30]{Ingo: #1}}
\newcommand{\todojr}[1]{\todo[inline, color=orange!30]{Jan: #1}}
\newcommand{\todomcp}[1]{\todo[inline, color=red!40]{Marie: #1}}


\begin{document}                     % und los!

	% ************************************************************
	% *****                    Titelseite                    *****
	% ************************************************************

	\setcounter{page}{1}
	\pagenumbering{Roman}

	\begin{titlepage}
	\thispagestyle{empty}
	{\center

			\vspace{1.5cm}
            {\LARGE  {\bf Story Diagrams -- Syntax and Semantics}\,
             \footnote{This work was partially developed in the Collaborative Research Center %developed in the course of the Special Research Initiative
             614, "Self-optimizing Concepts and Structures in Mechanical Engineering" at the University of Paderborn
             and was published on its behalf and funded by the Deutsche Forschungsgemeinschaft (DFG).}
             \footnote{This work was partially supported by the German Research Foundation (DFG) within the 
             Collaborative Research Centre ``On-The-Fly Computing'' (CRC 901).}
             \footnote{This work was partially developed in the project "ENTIME: Entwurfstechnik Intelligente Mechatronik" (Design Methods for Intelligent Mechatronic Systems).
             The project ENTIME is funded by the state of North Rhine-Westphalia (NRW), Germany and the European Union, European Regional Development Fund, "Investing in Your Future".}}
			\\
			\vspace{1.5cm}
			{\Large Technical Report} \\
			\vspace{1.25cm}
			 
			
			Markus von Detten, Christian Heinzemann, Marie Christin Platenius, \\
			Jan Rieke, Julian Suck, and Dietrich Travkin \\
			Software Engineering Group\\			
			Heinz Nixdorf Institute\\
			University of Paderborn\\
			Zukunftsmeile 1\\
			D-33102 Paderborn, Germany\\
			$[$Markus.von.Detten|Christian.Heinzemann|Marie.Christin.Platenius|\\
			Jan.Rieke|Julian.Suck|Dietrich.Travkin$]$@uni-paderborn.de\\

            \vspace{1cm}

			Stephan Hildebrandt\\
			Department System Analysis and Modeling\\
			Hasso-Plattner-Institut\\
			Prof.-Dr.-Helmert-Str. 2-3\\
			D-14482 Potsdam, Germany\\
			stephan.hildebrandt@hpi.uni-potsdam.de\\

            \vspace{1.5cm}

            Version: 0.1

			\vspace{2cm}
			Paderborn, \today

		}
	\end{titlepage}


	
%
%	 ~
%	\newpage
	
%	\input{content/eid.tex}
        
%	\newpage
%	~
% 	

% 	
% 	\newpage
% 	~	
	
	\tableofcontents
	\listoftodos
	\cleardoublepage
% 	~
% 
% 	\newpage
% 	~

% 	\newpage

	\pagenumbering{arabic}
	\setcounter{page}{1}

	\chapter{Introduction (Jan)}
% Problem Area
To tackle the complexity of modern technical systems, the engineering of such systems is heavily based on models.
Models are considered first-class artifacts of the development.
They describe different parts of the system in development, from different viewpoints and on different abstraction levels.
For instance, models are used to describe the structure and behavior of a software system, improving the overall comprehensibility of the system.
These models can then be employed to automatically generate code, reducing the risks of implementation errors.

As models evolve during the development process and different models have to be translated into each other and kept consistent, model transformations are a crucial part such a development process.
Model transformations are used to define in which way models can be changed, e.g., to specify refactoring operations.

Furthermore, model transformations itself can be employed to precisely specify the behavior of a system at run-time.
If, for example, a system should react to environment changes by reconfiguration, these reconfigurations can be described by model transformations which define how to reconfigure the system.
They furthermore allow a formal analysis, e.g., to prove that certain properties still hold after applying a transformation.

Story diagrams~\cite{ZSW99,FNTZ00,Zun01} are a powerful visual formalism for specifying model transformation, based on the well-known concept of graph transformation systems.
They feature declarative parts to specify object patterns which are matched and altered in the source model and combine them with an imperative part to specify the control flow of the transformation execution.
The concrete syntax of story diagrams extends the concrete syntax of UML activity diagrams.

% Problem
Since their introduction in 1998, several extensions to story diagrams have been proposed.
Furthermore, some semantic issues have been identified in the original concept~\cite{TMG06}.
In addition, the main story diagram tool, the Fujaba tool suite, has undergone major redesigns the last years; these redesigns also affected the story diagram implementation.
Moreover, new approaches like the Story Diagram Interpreter have emerged.

% Approach
In this technical report, we seek to provide a complete reference of the syntax and semantics of story diagrams.
It consolidates previous publications in a single document.
We provide definitions for the abstract and the concrete syntax as well as the semantics of story diagrams.

%Evaluation
As an example, we show how story diagrams can be used to specify refactoring operations on structural software models like class diagrams.

% Structure
\todojr{Struktur} 

	\chapter{Foundations}
\label{sec:foundations}

\todoall{What else do we need to address in this chapter?}

\section{Graphs and Graph Transformations}
\label{sec:foundations:simpleGTS}

Graphs consist of nodes and edges where an edge always connects two nodes. Nodes are used to represent objects and edges denote relationships between these objects. In course of this document, we will assume edges to be directed, i.e., they have a source node and a target node. In the most simple case, neither nodes nor edges have a predefined semantics \cite{Roz97}.

\begin{figure}[htbp]
  \centering
  \includegraphics[scale=1.5]{figures/SimpleGraph}
  \caption{Simple Graph}
  \label{fig:simpleGraph}
\end{figure}

Figure \ref{fig:simpleGraph} shows an example of a simple graph with three nodes and four edges. The nodes are visualized a circles, the edges are visualized as arrows. An edge may have the same node as a source and target node. Such an edge is called a \emph{self-edge}.

\emph{Graph transformation rules} specify allowed modifications of graphs. They consist of a left-hand side (LHS), a right-hand side (RHS), and a so-called rule morphism. Both, the LHS and the RHS are graphs while the rule morphism specifies which nodes of the LHS and RHS are considered to be the same. This information is required for the application of a graph transformation rule to a graph.

\begin{figure}[htbp]
  \centering
  \includegraphics[scale=1.5]{figures/SimpleGTRule}
  \caption{Simple Graph Transformation Rule}
  \label{fig:simpleGTRule}
\end{figure}

Figure \ref{fig:simpleGTRule} shows an example of a graph transformation rule. The LHS contains only node node with a self-edge. The RHS contains two nodes connected by an edge where the right node of the RHS has a self-edge as well. The rule morphism is visualized by the grey, dotted arrow. It specifies that the node of the LHS and the left node of the RHS are considered to be the same.

The application of a graph transformation rule to a graph is called a \emph{graph transformation} \cite{EEPT06}. The graph on which the rule is to be applied is called the \emph{host graph}.
The application of a graph transformation rule to a graph is performed in three steps. In the first step, an occurrence of the LHS of the graph transformation rule in the host graph is searched. Such an occurrence is called a \emph{match} of the graph transformation rule. If a match has been found, all nodes and edges that occur in the LHS but not in the RHS are deleted from the host graph. In this step, the rule morphism is used to decide which nodes do not occur in the RHS. In the third step, all nodes and edges that occur in the RHS but not in the LHS are added to the host graph. After the application of the graph transformation rule, there exists a match of the RHS into the host graph.

\begin{figure}[htbp]
  \centering
  \includegraphics[width=\linewidth]{figures/GTApplication}
  \caption{Application of a Graph Transformation Rule}
  \label{fig:GTApplication}
\end{figure}

Figure \ref{fig:GTApplication} shows an example of a graph transformation that applies the graph transformation rule of Figure \ref{fig:simpleGTRule} to the graph of Figure \ref{fig:simpleGraph}. The matching of the LHS into the host graph is visualized by a gray, dotted line. Then, the graph transformation rule deletes the self-edge from this node. Afterwards, a new node with a self-edge is created and connected to the previously matched node by an edge. The match of the RHS into the host graph after the rule application is again shown by gray, dotted lines.

In the field of algebraic graph transformations, the two most popular approaches for applying a graph transformation rule to a graph are the
\emph{double-pushout approach} \cite{Roz97} and the \emph{single-pushout
approach} \cite{Roz97}. The definition of story diagrams follows the
single-pushout approach. Besides the more theoretical differences the two
approaches differ in the handling of two special situations that might occur
upon rule application.

The first situation is the following. Assume the left-hand side of a rule
consists of two nodes. The first node is to be deleted and the second one is
to be preserved. Both of these nodes may be matched to the same node in the host
graph. In this situation, it is not clear if the node in the host graph is to be
deleted or preserved. The double-pushout approach explicitly forbids the application of the rule in such
situations. The single-pushout approach allows such situations and gives
deletion priority over preservation.

The second situation deals with dangling edges. It occurs if a certain node is
to be deleted but some of its incident edges are to be preserved. The
transformation would lead to a non-valid graph in which the edges would not have
either a source or a target node. The double pushout approach does not allow
such situations and instead requires that incident edges are explicitly
deleted. The single-pushout approach allows such situations and implicitly
deletes edges if one of the source or target nodes are deleted.

In general, matches of graph transformation rules are homomorphisms of the LHS of the rule to the host graph. That allows to match two nodes of the LHS to the same node of the host graph leading to first situation mentioned above. Such situations may be prevented by using isomorphisms for matching the LHS. Then, each node of the LHS must be matched to a unique node of the host graph. Thus, using isomorphic matchings solves the first situation when using single pushouts.

\section{Typed Attributed Graph Transformations}
\label{sec:foundations:typedAttrGTS}

Graphs and according graph transformations as introduced in Section \ref{sec:foundations:simpleGTS} are a very basic approach to modeling behavior. When using graph transformations for modeling behavior for object-oriented software or as a foundation for defining the semantics of modeling languages, it is necessary to distinguish different types of nodes and edges in a graph in order to give them semantics. 

Therefore, story diagrams are based on typed attributed graph transformations \cite{EEPT06}. Typed attributed graph transformations introduce a type graph and node attributes. The type graph defines different types of nodes and edges and it defines which types of edges are allowed for which type of nodes. Additionally, nodes may carry attributes like, e.g., objects in an object-oriented programming language. Accordingly, the type graph specifies inheritance relations between types of objects which are also known from object oriented programming languages.

If the graph transformation transforms a model based on a given type graph into a model based on the same type graph, the transformation is called \emph{endogenous}. Otherwise, i.e., the transformation transforms a model based on a given type graph into a model based on another type graph, the transformation is called \emph{exogenous.} Story diagrams are endogenous graph transformations.

\section{Type Graph used in this Report}
\label{sec:typeGraph}

\begin{figure}[htbp]
  \centering
  \includegraphics[width=\linewidth]{figures/gast-mm}
  \caption{Type graph of a generalized abstract syntax tree (GAST).}
  \label{fig:gast-mm}
\end{figure}

The type graph used in the examples in this report describes the structure of an abstract syntax tree. In particular it is an updated and slightly simplified version of the generalized abstract syntax tree (GAST) meta model developed in the QBench project \cite{QBench}. The GAST was developed to provide a unified syntax tree model for different programming languages like Java, C, and C++.
Figure~\ref{fig:gast-mm} shows an excerpt of that meta model. Especially some specialized sub classes have been omitted for clarity reasons.

\begin{description}
\item[Root] The \fe{Root} element is the central element of every GAST model. All other elements are reachable from the \fe{Root} node via composition relations.
\item[File] Elements of the GAST, e.g., classes and packages, can be assigned to files in the file system. A \fe{File} element holds references to those classes and packages and a String containing the path to the file.
\item[Package] Similar to packages in Java, the \fe{Package} element provides name spaces and visibilities. A \fe{Package} element can contain other packages, classes, global variables, and functions.
\item[GASTType] The \fe{GASTType} element represents data types like primitive data types and classes. The attribute \fe{qualifiedName} contains the unique, fully qualified name of the type.
\item[GASTClass] Classes are represented by the element \fe{GASTClass} in the GAST and are a sub type of the \fe{GASTType}. A \fe{GASTClass} holds references to its methods, attributes, and inner classes. A \fe{GASTClass} can be assigned to a \fe{Package}.
\item[Function] \fe{Function} is the super type for all executable operations. In addition to a name attribute, a \fe{Function} can have a number of local variables and formal parameters. The return type of a \fe{Function} is determined by its \fe{DeclarationTypeAccess}, a sub class of \fe{Access} (not shown in Figure~\ref{fig:gast-mm}). A \fe{Function} always contains a block statement which, in turn, can contain other statements.
\item[GlobalFunction] A \fe{GlobalFunction} element represents a globally accessible operation, i.e., an operation that does not belong to a class. They can be assigned to a name space defined by a package. For example, C functions are represented by \fe{GlobalFunctions}.
\item[Method] Functions that belong to an class are represented by \fe{Method} elements, a sub type of \fe{Function}.
\item[Variable] \fe{Variable} is a super type for all kinds of variables. A \fe{Variable} always has a name and a type.
\item[LocalVariable] \fe{LocalVariables} are variables that are contained in a \fe{Function}.
\item[FormalParameter] \fe{FormalParameters} are variables that represent the parameters of a \fe{Function},
\item[GlobalVariable] \fe{GlobalVariables} are variables that are globally accessible within a given scope. The scope is determined by the package in which the \fe{GlobalVariable} is contained.
\item[Field] The \fe{Field} element represents class variables. Therefore it is contained in a \fe{GASTClass}.
\item[Statement] A \fe{Function} consists of a number of \fe{Statements}. There are multiple sub classes of \fe{Statement} which represent the different kinds of statements. Most of them are omitted here. A \fe{Statement} can contain a number of \fe{Accesses}.
\item[BlockStatement] The \fe{BlockStatement} is a special kind of statement which can contain other \fe{Statements}. It is the root element of all \fe{Statements} contained within a \fe{Function}.
\item[Access] An \fe{Access} represents the use of a \fe{Variable} or a \fe{Function}. It always belongs to a certain \fe{Statement}.
\item[FunctionAccess] A \fe{FunctionAccess} represents the use of a \fe{Function} in a \fe{Statement} and therefore references the accessed \fe{Funtion} element.
\item[VariableAccess] A \fe{VariableAccess} represents the use of a \fe{Variable} in a \fe{Statement} and therefore references the accessed \fe{Variable} element.
\end{description}


	\chapter{Concepts} \label{sec:Concepts}

This chapter presents the concepts used in story diagrams. It begins with a short presentation of the general ideas of story diagrams and story patterns. Section~\ref{sec:StoryPatterns} then proceeds to explain story patterns in detail. It is followed by Section~\ref{sec:StoryDiagrams} which describes how story patterns are used in story diagrams and also presents calls between different story diagrams. Section~\ref{sec:Expressions} briefly covers the expressions that can be used in story diagrams.

	\section{Story Diagrams and Story Patterns in a Nutshell (Jan and Dietrich)} \label{sec:Overview}

%- model-driven software development, raise abstraction level and use models instead of code as the key artifact
%- specify structure and behavior of the software under development, make the models runnable/executable
%- UML offers notations for description of software structure and development, esp. class and activity diagrams
%- activity diagrams are too informal to be automatically executed (natural language used)
%- we replaced the informal activity descriptions with formal descriptions of operations on object structures and developed a new formal language called story diagrams

%- story diagrams are special UML activity diagrams
%- developed to formalize the description of a software's behavior (UML activity diagrams usually use informal textual descriptions of the tasks to be performed)
%- motivation: complete specification of a software, structure and behavior, i.e. make the software specification executable (code generation and interpretation)

In model-driven software development, a software model is the key artifact of the development.
It describes the structure of the software as well as its behavior and 
can be translated into executable source code or be interpreted to be executed.
The UML offers notations for the description of the software structure and behavior,
besides others class diagrams and activity diagrams.
However, UML activity diagrams use natural language to describe the particular activities, so they are not automatically executable.
Thus, a formal behavioral specification is needed.
For that purpose \emph{story diagrams} have been developed \cite{FNTZ00}.
They refine UML 1.5 \cite{UML1.5} activity diagrams and replace the natural language with a formal language to specify behavior
and, thus, can be automatically executed.

%- motivation: (formally) describe modifications of object structures for object-oriented software systems
%- use a graphical notation to specify operations on object structures (object structure modifications), OO world
%- each operation describes a modification of a given object structure, basically the modifactions are creations and removals of objects and their interconnections
%- graphically describe the object structure to be modified, mark the elements to be created and those to be removed

Story diagrams describe the control flow similar to UML activity diagrams by means of activity nodes and activity edges.
The behavior of each activity node is described using a graph transformation language called \emph{story patterns}.
Each activity node embeds one story pattern.
A story pattern uses a graphical notation to specify modifications of object structures in object-oriented software systems.
The modifications are basically creations and removals of objects and their interconnections (links).

%- motivation: use an appropriate, familiar, and simple notation for object structure modifications; we use a notation similar to UML object diagrams
%- motivation: declaratively describe the operations in activity nodes, thus, reduce complexity (avoid describing how to perform the operations)
%- motivation: keep determinism to a certain extent to specify the conditions for and the order of object structure modifications

To use a simple and familiar notation, story patterns are similar to UML object diagrams.
A story pattern represents an object structure to be modified and includes annotations specifying which objects and links are to be removed and created.
Story patterns are a declarative language, since they only specify what to remove and create, but not how and in which order.
This way, the complexity of the behavioral specifications is reduced.
In contrast to the deterministic control flow specified by activity nodes and edges that determine the order of story pattern executions,
the order of creations and deletions specified by a story pattern is non-deterministic.

%- motivation: base the specification on a well-known formalism (for execution and analyses)
%- story diagrams use graph transformations in their activity nodes (well-known formalism, exhaust the given theories for analyses and execution)

Story patterns base on \emph{graph transformation systems}, a well-known formalism and corresponding theories \cite{Roz97}.
Thus, precise analyses of the operations described by story patterns are possible,
e.g., it can be checked if certain properties of the object structure to be modified remain after the structure's modification.

%- given a so called host graph (an object structure or model), story diagrams describe the graph's modifications by means of creating or removing nodes and edges (objects and links)
%- the host graph, in our case, is a typed attributed graph, i.e. we have a graph to be modified (object structure, token model) and a corresponding type graph (type model or meta-model) describing the types and properties of the objects in our host graph
%- a graph transformation is executed by identifying a subgraph in the host graph which corresponds to the graph specified in the transformation (matching, subgraph isomorphism), removing nodes and links that are marked to be removed, and creating new nodes and links that are marked to be created

A story pattern specifies a graph transformation \cite{Roz97}.
Given a so-called \emph{host graph}, the graph to be modified, a graph transformation removes and creates nodes and edges in the given host graph.
The host graph is a typed attributed graph, i.e. there is a type graph determining the types and attributes of the nodes.
A graph transformation is executed on a host graph by identifying a subgraph in the host graph that is similar to the one specified in the graph transformation and then removing and adding the specified nodes and edges.
The identification of the subgraph for modification is called \emph{graph matching} and includes the \emph{subgraph isomorphism} problem.

In case of story patterns, the host graph is the object structure or model to be modified (\emph{token model}) \cite{Kue06},
i.e. the run-time data of the executed software.
The type graph is a set of classes and their relations which define all potential object structures at run-time.
These classes and relations are a so-called \emph{type model} or meta-model \cite{Kue06}.
They are required to specify story patterns.

For example, the class diagram in Figure~\ref{fig:SDExampleClassDiagram} defines the types \fe{Class} and \fe{Attribute} as well as their relations, attributes, and operations.
The corresponding story diagram in Figure~\ref{fig:SDExampleStoryDiagram} definies the behavior of the \fe{removeAttribute} method defined in the class diagram.
Here, the story diagram specifies that a class's attribute with the name given by the parameter \fe{text} is to be found in the class and in case of success this attribute is to be removed (\destroy).

\begin{figure}[htb]
	\centering
  \begin{minipage}[t]{.4\textwidth}
    \centering
    \includegraphics[scale=1]{SimpleSDRemoveAttributeClassDiagram} 
    \caption{Exemplary Type Model}
    \label{fig:SDExampleClassDiagram}
  \end{minipage}%
  \hfill
  \begin{minipage}[t]{.55\textwidth}
    \centering
    \includegraphics[scale=1]{SimpleSDRemoveAttribute}
    \caption{Exemplary Story Diagram}
    \label{fig:SDExampleStoryDiagram}
  \end{minipage}
\end{figure}

In summary, a story diagram is a special, formally defined UML activity diagram
that embeds graph transformations, so-called story patterns, in its activity nodes
to precisely describe run-time behavior by means of graph transformations.

%\subsection{Application scenarios (?)} \label{sec:Applications}

	\section{Story Patterns (Christian und Marie)} \label{sec:StoryPatterns}

In this section, we introduce story patterns in more detail. We start by giving the general idea of story pattern in Section \ref{sec:StoryPatterns:storyPattern}. Thereafter, we describe the basic concepts of story pattern, namely object variables, link variables and there respective binding semantics in Sections \ref{sec:StoryPatterns:objects} to \ref{sec:StoryPatterns:binding}. Finally, we show the use of object attributes in a story pattern in Section \ref{sec:StoryPatterns:attributes}.


\subsection{Story Pattern [CH]}
\label{sec:StoryPatterns:storyPattern}

Story patterns are typed attributed graph transformation rules with inheritance (cf. Section \ref{sec:foundations:typedAttrGTS}) that may be embedded into an activity node of a story diagram (cf. Section \ref{sec:StoryDiagrams}). Using a type model as introduced in Section \ref{sec:typeGraph} enables use of inheritance and polymorphism for matching nodes and edges.
This allows for specifying graph replacement rules for object-oriented models.

Story patterns are applied according to single pushout semantics using isomorphic matchings.
For enabling a concise notation of the graph transformation, story pattern apply a short-hand notation depicting the left-hand side (LHS) and the right-hand side (RHS) in a single, annotated graph. Nodes and edges being created (or deleted) are annotated with \create (or  \destroy, respectively).

Figure \ref{fig:simpleStoryPattern} shows an example of a single story pattern that creates a new method declaration in an interface and delegates the target of a call to this new method.

\begin{figure}[htbp]
  \centering
  \includegraphics[scale=1.0]{figures/SimpleStoryPattern}
  \caption{Simple Story Pattern}
  \label{fig:simpleStoryPattern}
\end{figure}

In the example, the nodes \fe{call}, \fe{interface}, and \fe{method} are bound objects (cf. Section \ref{sec:StoryPatterns:binding:states}), i.e., they are already matched. When applying the story pattern, the link from \fe{call} to \fe{method} will be deleted. Thereafter, the node \fe{methodDecl} of type \fe{Method} will be created along with the three links pointing to it. Upon creation, the attributes \fe{visibility}, \fe{abstract}, and \fe{name} are initialized with the specified values.

In a story pattern, we call all nodes and links that do not carry an annotation the \emph{core graph} of the story pattern. Then, the LHS of the story pattern consists of the core graph and all nodes and links annotated with \destroy. Accordingly, the RHS of the story pattern consists of the core graph and all node and links annotated with \create.

\subsection{Objects and Object Variables [MCP]}
\label{sec:StoryPatterns:objects}

Object variables represent the objects defined in a story pattern.
They are identified by their name.
The objects are instances of classes of the underlying classmodel (cp. Section \ref{sec:foundations:typedAttrGTS}).
Thus, the object variables are typed by classes from this model, i.e.
EClassifier objects.

The story pattern in Figure \ref{fig:simpleStoryPattern} contains four
object variables with the names \fe{call}, \fe{interface}, \fe{methodDecl}, and
\fe{method}. 
The type of an object variable is only visualized if the
variable is unbound (cf. Section \ref{sec:StoryPatterns:binding:states}). For
example the object variable \fe{methodDecl} has the type \fe{Method}.

Object variables have binding states, binding operators and binding semantics.
They are described in Section  \ref{sec:StoryPatterns:binding}.

\todomcp{Primitive Variables: concrete syntax like
object variables; binding expressions for initialization, see figure
\ref{fig:primitiveVariable}; primitive variables are typed over EDataType; they exist beyond the end of the Activity}

\begin{figure}[htbp]
  \centering
  \includegraphics[scale=0.6]{figures/PrimitiveVariable}
  \caption{Primitive variable with value assignment}
  \label{fig:primitiveVariable}
\end{figure}


\subsection{Links and Link Variables [MCP]}
\label{sec:StoryPatterns:links}

Link variables are used to connect different object variables. 
A link variable is typed over an EReference object of the underlying
classmodel.

Like object variables, link variables also have binding states, binding
operators and binding semantics (cf. Section \ref{sec:StoryPatterns:binding}).

\todomcp{binding state (unbound/bound) of link variables?}

\todomcp{Links to primitive variables: special LinkVariable, typed over
EStructuralFeature}

\subsection{Binding of Variables [MCP]}
\label{sec:StoryPatterns:binding}

Object variables and link variables have binding states (unbound, bound, maybe
bound), binding semantics (mandatory, negative, optional), and binding operators
(check only, create, destroy).


\subsubsection{Binding States}
\label{sec:StoryPatterns:binding:states}
An object variable or a link variable can either be bound, unbound, or maybe
bound. This is defined by the binding state. 
An unbound variable will be matched during the execution of the containing story
pattern. 
Even if the variable was already matched earlier in the story diagram, the
pattern matching algorithm is forced to find a new object for this variable.
A bound variable must have been matched previously. 
For a variable that is specified as maybe bound, a new match will only be
determined if it has not been bound before. 
Otherwise it will be treated as a bound variable.
The variable scope is limitted by the surrounding activity. 
That means a bound variable keeps its binding until the activity ends, unless
it has been matched again in the meantime.
An exception is the variable scope inside an each activity node (cf. Section
\label{sec:StoryDiagrams:composition}).

Unbound objects are visualized with a label of the form ``name:type'', while
the type is hidden for bound objects.

In a valid, interpretable story pattern, each connected component must contain
at least one bound object variable.

\todomcp{bindings states for links}

\todomcp{concrete syntax of maybe bound objects?}

\subsubsection{Binding Semantics}
\label{sec:StoryPatterns:binding:semantics}
Furthermore, object variables and link variables have binding semantics that
determine if a variable is mandatory, negative or optional.
Mandatory variables must exist in the given host graph, otherwise the pattern
fails. 
In contrast, negative variables must not exist in the host graph. If a variable
defined as negative can be matched during the execution of the story pattern,
the pattern fails, too.
Optional variables may exist. If an optional variable exists, it will be
bound, but if it does not exist, the story pattern may also be matched
successfully.

Negative variables are visualized crossed-out (cf. Figure
\ref{fig:bindingSemanticsOverview} b) and optional variables are visualized with
a dashed border (cf. Figure \ref{fig:bindingSemanticsOverview} c).
The same holds for negative and optional link variables (cf. Figure
\ref{fig:bindingSemanticsOverview} e+f).

Negative and optional object variables and link variables are not part of a
connected components. Regarding the rule that each connected component must contain
at least one bound object variable (cp. Section
\ref{sec:StoryPatterns:binding:states}), there are situations in which the
application of negative or optional object variables is not allowed. 

Figure \ref{fig:negativeObjects} shows example situations for negative objects
variables. 
For example, case a is allowed, but case b not because in the latter case the
graph is not connected.
The cases c is allowed because the object variables \fe{a} and \fe{c} are bound,
which means that each connected component has at least one bound object variables.
According to this, case d is allowed, too because \fe{a} and \fe{b} are both
bound. The cases e and f are allowed.

Similar to the application of negative object
variables, Figure \ref{fig:optionalObjects} shows some examples for the application of optional object variables. While case a is allowed, case b is not allowed because in this case the graph is not connected any more.
However, case c and d are allowed because each connected component contain at
least one bound object variable.
Case e and f are allowed.

\begin{figure}[htbp]
  \centering
  \includegraphics[scale=1.2]{figures/BindingSemanticsOverview}
  \caption{Binding Semantics for object and link variables}
  \label{fig:bindingSemanticsOverview}
\end{figure}

\begin{figure}[htbp]
  \centering
  \includegraphics[scale=1]{figures/negativeObjects}
  \caption{Negative Application Conditions}
  \label{fig:negativeObjects}
\end{figure}

\begin{figure}[htbp]
  \centering
  \includegraphics[scale=1]{figures/optionalObjects}
  \caption{Optional Object Variables}
  \label{fig:optionalObjects}
\end{figure}

\subsubsection{Binding Operators}
\label{sec:StoryPatterns:binding:operators}
Binding operators define whether an element or link is to be created, deleted,
or just matched.
After all elements that are defined to be deleted or just matched have been
matched, the model is modified by deleting and creating the elements as
defined.

Elements or links to be created are marked with the
label \create (cf. Figure \ref{fig:bindingOperatorsOverview} b+e) and elements
to be deleted are marked with the label \destroy (cf. Figure
\ref{fig:bindingOperatorsOverview} c+f).

\begin{figure}[htbp]
  \centering
  \includegraphics[scale=1.2]{figures/BindingOperatorsOverview}
  \caption{Binding Operators for object and link variables}
  \label{fig:bindingOperatorsOverview}
\end{figure}

\subsubsection{Feasible Binding Combinations}

Binding states, binding semantics and binding operators can be
arbitrarily combined, but only certain combinations are feasible. 
Table \ref{tab:bindingCombinations} shows which binding combinations may be
used with object variables.
As depicted there, bound and maybe bound object variables must not have negative
or optional binding semantics. As well, the combination of the binding states
bound or maybe bound and the binding operator create is not allowed.

% Table generated by Excel2LaTeX from sheet 'Tabelle1'
\begin{table}[htbp]
  \centering
  \caption{Feasible combinations of binding operators, binding states, and
  binding semantics for object variables}
    \begin{tabular}{|r|r|r|r|}
    \hline
    \textbf{Binding State} & \textbf{Binding Semantics} & \textbf{Binding
    Operator} & \textbf{Feasible} \\
    \hline
    UNBOUND & MANDATORY & CHECK\_ONLY & yes \\
    UNBOUND & MANDATORY & CREATE & yes \\
    UNBOUND & MANDATORY & DESTROY & yes \\
    UNBOUND & NEGATIVE & CHECK\_ONLY & yes \\
    UNBOUND & NEGATIVE & CREATE & no \\
    UNBOUND & NEGATIVE & DESTROY & no \\
    UNBOUND & OPTIONAL & CHECK\_ONLY & yes \\
    UNBOUND & OPTIONAL & CREATE & yes \\
    UNBOUND & OPTIONAL & DESTROY & yes \\
    \hline
    BOUND & MANDATORY & CHECK\_ONLY & yes \\
    BOUND & MANDATORY & CREATE & no \\
    BOUND & MANDATORY & DESTROY & yes \\
    BOUND & NEGATIVE & CHECK\_ONLY & no \\
    BOUND & NEGATIVE & CREATE & no \\
    BOUND & NEGATIVE & DESTROY & no \\
    BOUND & OPTIONAL & CHECK\_ONLY & no \\
    BOUND & OPTIONAL & CREATE & no \\
    BOUND & OPTIONAL & DESTROY & no \\
    \hline
    MAYBE\_BOUND & MANDATORY & CHECK\_ONLY & yes \\
    MAYBE\_BOUND & MANDATORY & CREATE & no \\
    MAYBE\_BOUND & MANDATORY & DESTROY & yes \\
    MAYBE\_BOUND & NEGATIVE & CHECK\_ONLY & no \\
    MAYBE\_BOUND & NEGATIVE & CREATE & no \\
    MAYBE\_BOUND & NEGATIVE & DESTROY & no \\
    MAYBE\_BOUND & OPTIONAL & CHECK\_ONLY & no \\
    MAYBE\_BOUND & OPTIONAL & CREATE & no \\
    MAYBE\_BOUND & OPTIONAL & DESTROY & no \\
    \hline
    \end{tabular}%
  \label{tab:bindingCombinations}%
\end{table}%

\todomcp{see albert's habil for example for optional-create}
\todomcp{table for link variables and for object set variables?}

\subsection{Using Object Attributes [CH]}
\label{sec:StoryPatterns:attributes}

The objects of our host graph carries attributes. During the application of a story pattern, these attributes may be used twofold. First, constraints may be specified on the attributes to restrict the possible matches of a story pattern. Second, attribute values may be changed during the graph rewriting step.

In story pattern, we may restrict the matchings of nodes to objects of the host graph that have specific attribute values. We achieve this by so-called \emph{object constraints}. Such object constraints are considered to be part of the LHS and do not change the host graph. Figure \ref{fig:objectConstraint} shows an example.

\begin{figure}[htbp]
  \centering
  \includegraphics[scale=1]{figures/ObjectConstraint}
  \caption{Matching with an Object Constraint}
  \label{fig:objectConstraint}
\end{figure}

In the example, we match a method being contained the class \fe{class}. The match is restricted to a method which has a name that equals variable \emph{methodDecl.name}.

The values of attributes that are not restricted by an object constraint are not considered during the matching. Thus, they may have an arbitrary value. In the current version of story pattern, such object constraints need to be specified using OCL. Besides equality checks, all comparative operations on the attributes of the object supported by OCL may be used as object constraints. OCL statements, however, that traverse the references of the objects are not to be used but modeled as part of the story pattern.

Beside object constraints, we may also use \emph{attribute assignments} to change the value of an attribute during the application of the story pattern. Thus, attribute assignments are considered to be part of the RHS. When using attribute assignments, the value of the attribute is not considered while matching the LHS to the host graph. Figure \ref{fig:attributeAssignment} shows an example.

\begin{figure}[htbp]
  \centering
  \includegraphics[scale=1]{figures/AttributeAssignment}
  \caption{Using an Attribute Assignment}
  \label{fig:attributeAssignment}
\end{figure}

In the example, a method of class \fe{class} with an arbitrary name is matched. Then, the name of the method is changed to the name that equals variable \emph{methodDecl.name}. 

The concrete syntax of an attribute assignment is
\begin{lstlisting}
 <attributeAssignment> ::= #Attribute.name ':=' Expression
\end{lstlisting}
The expression is to be specified using OCL. The type of the return value of the OCL expression must be the same as the type of the attribute. Since the attribute assignment is considered to be part of the RHS, it is visualized in green color.


\subsection{Object Sets [MCP]}

explain objectSetVariables, set size expressions

\todomcp{object sets and binding operators/states/semantics}

\todomcp{If we bind an object set, can we use the bound object in other story pattern? E.g. to insert all elements bound by the object set into a container via a containment link? (See Figure \ref{fig:reuseObjSet}).}
\tododt{Yes, but I would use another concrete syntax (see Figures~\ref{fig:reuseObjSet1}, \ref{fig:reuseObjSet2}, and \ref{fig:reuseObjSet1}).}

\begin{figure}[p]
	\begin{minipage}{.45\textwidth}
		\centering
		\includegraphics[scale=.8]{figures/ReuseObjectSet}
  	\caption{Reusing Object Sets}
  	\label{fig:reuseObjSet}
	\end{minipage}
  \hfill
  \begin{minipage}{.45\textwidth}
  	\centering
		\includegraphics[scale=.8]{figures/ReuseObjectSet1}
  	\caption{Reuse objects in a set}
  	\label{fig:reuseObjSet1}
	\end{minipage}
\end{figure}

\begin{figure}[p]
	\begin{minipage}{.45\textwidth}
		\centering
		\includegraphics[scale=.8]{figures/ReuseObjectSet2}
  	\caption{Add an object to a set}
  	\label{fig:reuseObjSet2}
	\end{minipage}
  \hfill
  \begin{minipage}{.45\textwidth}
  	\centering
		\includegraphics[scale=.8]{figures/ReuseObjectSet3}
  	\caption{Add all objects from a set to a container}
  	\label{fig:reuseObjSet3}
	\end{minipage}
\end{figure}

\todomcp{An object set contains no ObjectSetSizeExpression and no object is matched into the object set: ObjectSet is interpreted as optional and the matching succeeds.}

\todomcp{All operators for comparison are allowed: <, <=, >, >=, = !=}


\begin{figure}[p]
	\begin{minipage}{.45\textwidth}
		\centering
		\includegraphics[scale=.8]{figures/ObjectSetSize}
  	\caption{Object Set Size}
  	\label{fig:objSetSize}
	\end{minipage}
  \hfill
  \begin{minipage}{.45\textwidth}
  	\centering
		\includegraphics[scale=.8]{figures/IsomorphismInObjectSets}
  	\caption{Isomorphism in Object Sets}
  	\label{fig:isoObjSet}
	\end{minipage}
\end{figure}

\todomcp{What happens if one pattern contains two object sets that may possibly contain the same objects. Consider the pattern in Figure \ref{fig:isoObjSet}. If c1 and c2 share the same super classes, su1 and su2 contain the same objects. Is this allowed? Isomorphic matching would normally forbid this.}

\tododt{I would allow this which would comply with our isomorphic matching.
But in this case it is non-deterministic how the objects are matched to the set nodes (assuming the classes are already bound).
A maybe constraint could allow to match the same objects to both sets.}
\todojr{We decided to match set nodes without them having to be disjoint, thus, we do not enforce isomorphism for the content of two or more set variables.}

\subsection{Special Link Kinds [CH]}

\subsubsection{Inclusion Links [CH]}

\todoch{What is the semantics of a containment link? Current understanding: an element is contained in a container. What is the difference to to-many references which are containments?}
\tododt{Containment links do not correspond to any association or reference. They only describe containments of objects in containers or set nodes.}

\todoch{Which classifiers are allowed for ContainerVariables? This should not only be the EMF collection types EList and EMap. Does a ContainmentLink need a reference as a type like normal LinkVariables?}
\tododt{The contaiment links do not need any reference. I would suggest to allow any subtype of java.util.Collection (and java.util.List in case that link order constraints are used), but not java.util.Map! This is something different and needs a key for containment. Maybe we need special containment links for maps that include a key.}

\subsubsection{Paths [MCP]}

\subsubsection{Link Constraints [CH]}

\todoch{What is the concrete Syntax for this?}

\tododt{I have a suggestion in Figure~\ref{fig:linkConstraints}.}

\begin{figure}[p]
  \centering
  \includegraphics[scale=.8]{figures/LinkConstraints1}
  \caption{Link order constraints FIRST and DIRECT\_SUCCESSOR}
  \label{fig:linkConstraints}
\end{figure}

Link constraints are only applicable to link variables that reference an ordered to-many reference.
\todoch{No link constraints for inclusion links.}

\begin{itemize}
  \item FIRST = matches the first element in the list, requires one link variable
  \item LAST = matches the last element in the list, requires one link variable
%  \item INDEX = matches the element at the specified index, requires one link variable
%  \todoch{Our lowest index value is 0, not 1.}
  \item DIRECT\_SUCCESSOR = requires two link variables, target of the second one must be located directly after the target of the first one in the list
  \item SUCCESSOR = requires two link variables, target of the second one must be located somewhere after the target of the first one in the list
\end{itemize}

\todojr{We will separate link constraints for single links and those for two links and their order. Stephan will make a suggestion.}

\todoch{Is this semantic description correct/complete?}

\todoall{Stephan: Following is my suggestion for link constraints.}

Link constraints specify constraints on the absolute position of an element in an ordered reference and the position of an element relative to another element. These constraints are only applicable to \emph{LinkVariables} that refer to ordered multi-valued references. Other kinds of links cannot be adorned with link constraints.

\paragraph{Link Position Constraints}

\begin{figure}[htb]
\center
\includegraphics[width=0.75\columnwidth]{figures/linkPositionConstraint1.png}
\caption{A \emph{first} link position constraint}
\label{fig:linkPositionConstraints:linkPositionConstraint1}
\end{figure}

A link position constraint specifies that the target object of the constrained link has to be the \emph{first} or the \emph{last} element in that reference. For example, in a valid match of the story pattern in Fig.~\ref{fig:linkPositionConstraints:linkPositionConstraint1}, the object matched to \emph{bx} must be also be the first element in the \emph{elements} reference of the object matched to \emph{a}. Link position constraints support only these two positions: \emph{First} and \emph{last}. If a story pattern contains multiple links that refer to the same metamodel reference and have the same source variable, only one of them may have a \emph{first} (or \emph{last}) constraint.

When link position constraints are attached to links with \destroy stereotypes, the semantics are identical, even if they are optional. For links with \create stereotype, the link's target element is inserted at the specified position. For optionally created links, the semantics is somewhat different. If the link does not yet exist, it is created. The target element is inserted at the specified position in the reference. If the link does exist but the element is not at the specified position, it is moved.

\todoall{Alternatives: 1.) Ignore link position constraints on optional-create links, 2.) add target element a second time. For unique references, this has the same effect like moving the element, in case of non-unique references, the element exists twice (or more times) in the list, afterwards.}

\begin{figure}[htb]
\center
\includegraphics[width=0.75\columnwidth]{figures/linkPositionConstraint2.png}
\caption{A negative link position constraint}
\label{fig:linkPositionConstraints:linkPositionConstraint2}
\end{figure}

A combination with negative links is also straight-forward. In a valid match of the pattern shown in Fig.~\ref{fig:linkPositionConstraints:linkPositionConstraint2}, the match of \emph{bx} must \emph{not} be the first element in the \emph{elements} reference.

\paragraph{Link Order Constraints}

\begin{figure}
\center
\includegraphics[width=0.4\columnwidth]{figures/linkOrderConstraint1.png}
\caption{A link order constraint specifies a relative order between two links.}
\label{fig:linkOrderConstraints:linkOrderConstaint1}
\end{figure}

Link order constraints specify a relative order between two links. In Fig.~\ref{fig:linkOrderConstraints:linkOrderConstaint1}, \emph{b2} must be the immediate successor of \emph{b1} in the \emph{elements} reference (depicted by \emph{\{next\}}). It is also possible to specify that \emph{b2} may be a direct or indirect successor, which is depicted by \emph{\{...\}}. Note, that then there may be multiple possible matches. Which of them is chosen is still non-deterministic.

Link order constraints can be combined straight-forwardly with \create and \destroy stereotypes. However, the combination with \create is only possible for the \emph{\{next\}} order constraint, not for \emph{\{...\}}. In case of created links, the order in the instance model is enforced accordingly.

\todoall{Do we want to allow this? This would make the choice, where an element is inserted non-deterministic.}

It is also possible to have only one of the two related links be destroyed or created. Though, it is not possible to create one link and destroy the other.

Link order constraints may not form circles or unsatisfiable story patterns (e.g., where a link is the successor of another but also has a \emph{\{first\}} link position constraint).

\begin{figure}
\center
\includegraphics[width=0.4\columnwidth]{figures/linkOrderConstraint2.png}
\caption{A link order constraint with a negative link.}
\label{fig:linkOrderConstraints:linkOrderConstaint2}
\end{figure}

Link order constraints can also be combined with negative link. However, only one of both links may be marked as negative. Otherwise, there would be nothing to relate to. For example, in the story pattern in Fig.~\ref{fig:linkOrderConstraints:linkOrderConstaint2}, \emph{b2} may not be the direct successor of \emph{b1}.

\begin{figure}
\center
\includegraphics[width=0.6\columnwidth]{figures/linkOrderConstraint3.png}
\caption{Two link order constraints with a negative link.}
\label{fig:linkOrderConstraints:linkOrderConstaint3}
\end{figure}

If there are multiple link order constraints, like in Fig.~\ref{fig:linkOrderConstraints:linkOrderConstaint3}, the semantics of the negative link is as follows: \emph{b2} must neither be the direct successor of \emph{b1} nor the direct predecessor of \emph{b3}. If one of both conditions is fulfilled, the pattern does not match.

\todoall{This should match the semantics of multiple NACs in a story pattern.}

For optionally created and destroyed links, the same applies to link order constraints like to link position constraints. Optionally destroyed link can be used straight-forwardly. An optionally created link is created if it does not exist, so that the specified order is enforced. If the link does exist but the element is not at an appropriate position, it is moved.

\subsubsection{Maybe Links}

Disables the isomorphism check for two object variables, these two object variables may be matched to the same object.

\todoch{What is the concrete Syntax? Using a special pattern constraint as proposed in Alberts Habil is very low-level. Alternative version is proposed in Figure \ref{fig:maybeLink}.}
\todoch{conrete syntax: "same?"}

\begin{figure}[htbp]
  \centering
  \includegraphics[scale=.8]{figures/MaybeLink}
  \caption{Maybe Link}
  \label{fig:maybeLink}
\end{figure}

\subsection{Pattern Constraints [CH]}

\todoch{What happens when a pattern constraint is placed inside a for each story pattern (not inside a node in that pattern)? Proposal: The particular match must fulfill the pattern constraint, if it does not fulfill the pattern constraint, the match is rejected and the iteration continues.}
\tododt{Exactly. In this case, it is a kind of post condition that has to be satisfied at the end of the matching step.}

\subsection{Pattern Fragments [MCP]}

patterns contained in other patterns, negative, semantics? review enhanced story patterns from Diss Florian Klein

\todomcp{Should contained pattern be marked as forEach? Idea for semantics: first the part of the pattern outside the forEach pattern is matched, then the forEach subpattern is applied to any match that may be located, the variables bound in a forEach subpattern may not be used in subsequent activities}

\tododt{This is somewhat confusing.
As I understand them, subpatterns are ordinary story patterns within another story pattern.
They are surrounded by a fragment box and can be labeled with a name (see Figure~\ref{fig:labeledSubPattern}).
Special types of such subpatterns are negative application condition fragments (NACs), set fragments, and optional fragments.
As far as I know, we did not plan to add $\forall$ and $\exists$ fragments, did we?
These are only used in SDDs and TSSDs which are constraint languages.
}

\begin{figure}[htbp]
  \centering
  \includegraphics[scale=1.0]{figures/ContainedPattern}
  \caption{Different Kinds of Contained Patterns}
  \label{fig:containedPattern}
\end{figure}

\todomcp{Should contained pattern be marked as optional? Is currently possible in the meta-model. Idea for semantics: Whole pattern must be found, if found, variables may be used in subsequent activities, if pattern may not be found as a whole, matching still succeeds but all variables in the subpattern are not bound in subsequent activities.}
\tododt{I would say, contained patterns are mandatory in general (or are NAC/optional/set in case of the according fragment).}

\begin{figure}[htbp]
  \centering
  \includegraphics[scale=1.0]{figures/SubPatterns2}
  \caption{Labeled sub pattern}
  \label{fig:labeledSubPattern}
\end{figure}

\begin{figure}[htbp]
  \centering
  \includegraphics[scale=1.0]{figures/SubPatterns1}
  \caption{Hierarchies of NAC, set, and optional sub patterns}
  \label{fig:subPatternHierarchies}
\end{figure}

\todomcp{How deep may patterns be nested? What is the semantics of alternating binding semantics of sub-patterns, e.g. negative in optional in negative and so on.}
\tododt{I would prefer to allow arbitrarily deep nestings and would suggest to interpret the fragments in the order from outside to inside. Example (see Figure~\ref{fig:subPatternHierarchies}): You match a super class \emph{superClass} of \emph{myClass} and ensure that \emph{superClass} has no attribute. Then you you match all methods \emph{new} (outer set fragment) that have no class as their type (enclosed NAC fragment). Now you match for each of these methods all parameters (enclosed set fragment) that have \emph{myClass} as their type. Furthermore, you try to find a path from the matched \emph{new} method to a return statement (optional fragment).}



 
	
	\section{Story Diagrams} \label{sec:StoryDiagrams}

\subsection{General Idea (Dietrich)}

% - combine activity diagrams and graph transformations as well as imperative, deterministic and declarative, non-deterministic languages to formally and compactly describe software behavior in terms of model transformations using an OO-based, familiar notation

The main idea behind story diagrams is to formalize UML activity diagrams
to better support model-driven software development
by means of completely modeling software structure and behavior and making the software model executable.
For that purpose, graph transformations, a well-known formalism, were chosen to formally specify behavior and have been combined with UML activity diagrams.
The result, story diagrams, is a mixture of two languages:
an imperative, deterministic language for the description of control flow, UML activity diagrams,
and a declarative, non-deterministic, object-oriented, graph-transformation-based language for the description of model modifications, so-called story patterns (see Section~\ref{sec:StoryPatterns}).
Both languages are graphical, formally defined, and use a familiar notation based on UML activity diagrams \todoall{ver. 1.5 or 1.4?} and UML object diagrams with minor modifications.

An exemplary story diagram is illustrated in Figure~\ref{fig:simpleStoryDiagram}.
This story diagram replaces calls of a given method (\emph{oldMethod}) with calls of another given method (\emph{newMethod}).

\begin{figure}[htb]
  \centering
  \includegraphics[scale=1.0]{figures/SimpleStoryDiagramExample}
  \caption{Exemplary Story Diagram -- Replace Method Calls}
  \label{fig:simpleStoryDiagram}
\end{figure}

Like UML activity diagrams, story diagrams model control flow by means of activity nodes and activity edges.
Each activity node contains a story pattern to formally specify the behavior for this node\footnote{There are
some few exceptions like activity call nodes which do not contain story patterns to specify the behavior.}.
The activity edges can carry guards.
These are either boolean expressions, e.g., checking attribute values of a matched object,
or guards used to specify decisions on whether a story pattern could be
matched or not\footnote{A story pattern is successfully matched if for each object and link variable in the pattern corresponding objects and links are found in the host graph and all specified constraints are satisfied.}.
In Figure~\ref{fig:simpleStoryDiagram} the used guards are \text[success\text] (successful matching),
\text[failure\text] (failed to completely execute a story pattern),
and \text[end\text] (activity edge points to the first activity node to be executed after a loop).
In contrast to ordinary UML activity diagrams, story diagrams, so far, do not model concurrent execution.
Thus, the language constructs \emph{fork} and \emph{join} are omitted in story diagrams.

%- Story diagrams in MDSD process, 2 worlds: stand-alone transformations and specifications of methods' behavior:
%  1. alternative (completely modeling software): model classes in class diagrams, specify their methods' behavior in story diagrams, generate executable source code (e.g. Java) or use an interpreter
%  2. alternative (specify recurring model operations/transformations for a given type of models): model only classes representing the editor's model under development (meta-model), specify modification operations of this model (adding and removing elements, analysis operations, translations to/generation of other models, etc.), need of a software that triggers the specified operations, the operations can be performed using generated code or an interpreter
%- introduce an example

Basically, there are two slightly different ways of using story diagrams in a model-driven software development process.

Originally, story diagrams were used to formally specify the behavior of methods that are defined in classes in object-oriented software development.
Calling such a method would mean to execute the story diagram that represents the method's behavior.
If for all defined methods story diagrams are specified, the software can be executed.
In this case, story diagrams specify the behavior of objects whose properties are defined by classes.
Thus, story diagrams have a \emph{this} variable representing an object (class instance) that they belong to.
This variable can be used as a starting point for the graph matching specified in a story diagram.
\tododt{add an example}

Another more flexible way of using story diagrams is to specify any kind of model transformation or operation in a story diagram without sticking this behavior to a certain class.
In contrast to the previous case, there is no \emph{this} variable that could be used as a starting point for the graph matching.
All starting nodes for the graph matching have to be provided as arguments of the story diagram call.
For this purpose, the story diagram in Figure~\ref{fig:simpleStoryDiagram} has two parameters \fe{oldMethod} and \fe{newMethod}.
The corresponding arguments of a story diagram call are assumed to be known (bound object variables) and are used as starting points for the graph matching.
This way, the operations or transformations defined by story diagrams can be used from within any other part of the developed software, like a software library would be used.
Typically, model-to-model transformations, consistency checks, or more generally speaking, recurring and object-independent operations are defined this way.
\tododt{add an example}

% - enable formal analyses to check/ensure certain behavioral software properties

In both cases the story diagrams can be used to generate executable source code or be executed using an interpreter.
Besides execution, the formally defined story diagrams can also be analyzed to guarantee certain behavioral properties.
For example, model checking can be used to check if a safety property is always met (\todoall{example!!!}) or if a critical state (\todoall{example!!!}) can ever be reached.


%\subsection{The Language Constructs in Story Diagrams (Jan/Dietrich)}\label{sec:StoryDiagrams:composition}

\begin{itemize}
%  \item Combination of an excerpt of UML 1.5 activity diagrams and story patterns (incl. example)
%  \item list the major language constructs like activity nodes and edges, guards (concrete syntax, well-formedness, and execution semantics)
%  \item name our restrictions, e.g., omition of fork and join
  \item refer to the appendix due to a complete language reference
  \item mention the grammar for story diagrams (based on grammar in diploma thesis of Thomas Klein \cite{Kle99})?
\end{itemize}


\subsection{Activities, Activity Parameters and Return Values (Dietrich)}

Since story diagrams can be seen as special UML activity diagrams, we reused the class names defined by the UML.
Thus, similar to UML activity diagrams, a story diagram is represented by a so-called activity (class \fe{Activity}).

Each story diagram can have parameters.
We distinguish \emph{in} and \emph{out} parameters,
i.e. parameters representing arguments given when a story diagram is called (\emph{in})
and parameters representing return values (\emph{out}).
Parameters can be (\emph{in}) and (\emph{out}) parameter at the same time.
The story diagram in Figure~\ref{fig:simpleStoryDiagram} has two (\emph{in}) parameters \fe{oldMethod} and \fe{newMethod}
as well as an unnamed (\emph{out}) parameter of the type \fe{Call}.

If a story diagram defines the behavior of a method, the parameters are defined by the corresponding method's signature.
In this case, the number of \emph{out} parameters is limited to one single parameter and represents the only \emph{return} value of the method and story diagram.
Besides these parameters there is another implicitly defined parameter \emph{this}
which -- similar to Java's \emph{this} keyword -- represents the object that the story diagram belongs to.

In case that a story diagram is not defining a method's behavior, it defines it's own signature explicitly with according (\emph{in}) and (\emph{out}) parameters.
The number of \emph{out} parameters is allowed to be arbitrary in this case and there is no \emph{this} parameter.

The values or objects returned after execution of a story diagram are defined by the object variable listed in the \emph{stop} activity nodes.
For example, the object matched to the object variable \fe{c} is returned by the story diagram in Figure~\ref{fig:simpleStoryDiagram} in case of a successful execution.
Otherwise, the \emph{null} reference is returned which is specified by the keyword \emph{null}.


\subsection{Activity Nodes, Activity Edges (Dietrich)}

\todoall{Story Nodes (laut Doku im Meta Modell ist ein
Story Node ``an activity node containing a story pattern'') werden in Modifying
und Matching Story Nodes unterteilt. }

A story diagram's control flow is defined by activity nodes and activity edges, similar to UML activity diagrams.
Except of a few exceptions where an activity node represents a call of another story diagram,
each node contains a story pattern to specify the corresponding behavior.
Thus, executing an activity node results in executing the corresponding story pattern.

In contrast to single story patterns, story patterns contained in activity nodes of a story diagram have another scope.
Here, you can reuse all object variables declared in a story pattern of a preceding activity node.
For example, in Figure~\ref{fig:simpleStoryDiagram}, the object variable \fe{parentClass} is reused in the following story pattern
by specifying the variable as a bound variable
(i.e. the variable does not have to be matched anymore, the object matched in the steps before is used instead).

Executing an activity node means to execute the corresponding story pattern
which means to find a subgraph with the specified properties (e.g. find objects of a certain type and with certain connections).
As a consequence, since trying to find such a subgraph can fail, each execution of an activity node can fail.
To distinguish the cases of a successful activity node execution and its failure,
the outgoing activity edges can be provided with the guards \text[success\text] and \text[failure\text].
The control flow is following the activity edge with the \emph{success} guard in case of a successful activity node execution,
i.e. a successful matching of the corresponding story pattern,
otherwise it follows the activity edge with the \emph{failure} guard.
If an outgoing activity edge has no guard, it covers both cases, success and failure.
The guards \text[success\text] and \text[failure\text] can only be used pair-wise.
The first activity node in Figure~\ref{fig:simpleStoryDiagram} has two outgoing activity edges with these guards.

A special activity node is the \emph{for-each} activity node, depicted by a cascading activity node.
The second activity node in Figure~\ref{fig:simpleStoryDiagram} is a \emph{for-each} activity node.
Such a node represents a loop where the contained story pattern is executed as often as new subgraphs can be matched
that differ from the previously matched graphs by at least one other matched object.
For example, the story pattern in the \emph{for-each} activity node in Figure~\ref{fig:simpleStoryDiagram}
is matched for each existing pair of a method (object variable \fe{anyMethod}) and corresponding call object (object variable \fe{c}).
Besides the matching itself, all \emph{destroy} and \emph{create} steps are also executed for each of these matched subgraphs.
After all these matching steps, the control flow is guided by the outgoing activity edge with the guard \text[end\text].
Each \emph{for-each} activity node has such an outgoing activity edge.
In addition, this edge can be combined with another activity edge having the guard \text[for each\text].
This edge is leading to the activity node (or e.g. a sequence of such nodes)
that is to be executed after each successful execution of the \emph{for-each} activity node.
In this case, there has to be an activity edge leading back to the \emph{for-each} activity node
to terminate the specified loop by following the \text[end\text] activity edge.



\subsection{Decision Nodes and Guards (Jan)}
- no fork and join

\subsection{Loops (Jan)}

\subsection{Story Diagram Calls (Markus)}

Story diagram calls are special nodes in a story diagram which are used to invoke other story diagrams. Similar to method calls, this reduces redundancy and promotes reuse.

As described in the previous sections, a story diagram can have an arbitrary number of in and out parameters. When calling a story diagram, concrete arguments have to be assigned to the in parameters. Consequently, if an object variable named n is bound somewhere in the story diagram, the identifier n can be used to pass this object variable as an argument to a call. Similarly, if the called story diagram has out parameters, those are implicitly bound to their respective names and can be used by specifying object variables of the same name.

In contrast to QVT \cite{QVT}, we do not explicitly model inout parameters. Instead, we allow the same objects which are passed as in parameters to be also returned as out parameters.
These objects are not bound again after the call. Instead the binding is preserved and the same object variable is used before and after the call. This is similar to the semantics of a call-by-reference parameter.

An example of a story diagram call is shown in Figure~\ref{fig:call}.

\begin{figure}[htb]
\begin{center}
  \includegraphics[width=\textwidth]{figures/StoryDiagramCall}
  \caption{Example of a story diagram call}
  \label{fig:call}
\end{center}
\end{figure}

The first story pattern in Figure~\ref{fig:call}, shows the bound object variable \fe{package}. Two new object variables \fe{class1} and \fe{class2} are bound in that pattern. The next node with the grey background is a story diagram call which is also signified by its label. Beneath the label, the name of the called story diagram is given, in this case \fe{CreateBidirectionalAssociation}. Assume that the called story diagram has two in parameters of the type \fe{Class} and one out parameter of the type \fe{Association}. The two classes that were bound in the first story pattern, \fe{class1} and \fe{class2} are passed to the call as arguments.
The result of the call is bound to the object variable \fe{assoc}. The type of this variable is determined by the out parameter type, i.e., in this case the type Association.

If a story diagram has no out parameters, the colon and the out parameters names after the parentheses are omitted (see Figure \ref{fig:SDRemoveInterfaceViolation} for an example).

%Issues for future versions:
% method calls
% polymorphic calls

\ext{
\subsection{Exception Handling (maybe not in Ver. 0.1)}
}




	\section{Expressions (Julian and Dietrich)} \label{sec:Expressions}

\tododt{How can all the expressions be specified and visualized uniformly?}
	
	\input{03e_Concept_Templates.tex}




	
	\chapter{Complete Example (Markus)}

This chapter presents a complete example of a transformation with story diagrams. The setting of the example is explained in Section \ref{sec:Example:Motivation}. The next section then presents several complex story diagrams that the execute the transformation.

\section{Motivation of the example}
\label{sec:Example:Motivation}

One well-known principle of object-oriented programming says \emph{``Program to an interface, not an implementation.''} \cite{GHJV95}. By only accessing interfaces instead of concrete classes from a given class, that class remains independent of concrete implementations. The accessed class can be exchanged transparently without breaking the program. If this principle is neglected, accidentally or intentionally, this is known as an \emph{interface violation}.

\begin{figure}[hbtp]
\centering
\includegraphics[width=\linewidth]{./figures/InterfaceViolation}
\caption{Example of an interface violation}
\label{fig:InterfaceViolationExample}
\end{figure}

In Figure~\ref{fig:InterfaceViolationExample}, a simple example of an interface violation is depicted. The classes \fe{A} and \fe{B} implement the interfaces \fe{IA} and \fe{IB}, respectively. Following the design principle ``Program to an interface, not an implementation'', the classes are expected to communicate through their interfaces. However, \fe{A} calls the method \fe{m3()} from \fe{B}. Because \fe{m3()} is not provided by the interface \fe{IB}, \fe{A} downcasts the object \fe{ib} to the concrete type \fe{B} in order to access \fe{m3()}. This intentional bypassing of the interface \fe{IB} is an Interface Violation.

There are several possibilities to remove an interface violation from a program. A trivial solution would be to delete the downcast and the call from the implementation \fe{m1}. This would, of course, remove the interface violation but also change the program behaviour. A more sensible the solution, that will be used in this chapter is the extension of the interface \fe{IB} to contain the method declaration of \fe{m3}. By adding this declaration to \fe{IB}, the class \fe{A} can call \fe{m3} via the interface. The downcast becomes unnecessary and can be removed. At the same time, the behaviour of \fe{m1} is preserved.

To this point, the refactoring is very similar to the \emph{Extract Interface} refactoring described by fowler \cite{Fow99}. Extending an existing interface, however, is a little more complicated as there may already be other classes that implement \fe{IB}. If \fe{m3} is added to \fe{IB}, those other implementing classes all have to be extended by an (possibly empty) method implementation of \fe{m3} in order to remain compilable.

A story diagram that removes an interface violation by extending an interface as described above is presented in the following section.

\section{Story diagram: Remove interface violation}

\begin{figure}[hbtp]
\centering
\includegraphics[width=\linewidth]{./figures/SDRemoveInterfaceViolation}
\caption{Story Diagram: RemoveInterfaceViolation}
\label{fig:SDRemoveInterfaceViolation}
\end{figure}

Figure~\ref{fig:SDRemoveInterfaceViolation} shows the story diagram to remove an interface violation. It consists of five story activities and two activity calls. This section explains the story diagram step by step.

The story diagram has five in-parameters: \fe{call}, \fe{interface}, \fe{method}, \fe{castStmt}, and \fe{accessedMethodOwner}. The \fe{call} represents the statement that calls the method in the concrete class (the call of \fe{m3} in \fe{m1}). The \fe{interface} is the interface that will be extended (\fe{IB} in the example). \fe{method} is the method that is currently not declared in the interface (\fe{m3()}). \fe{castStmt} refers to the statement that down casts the interface type to the concrete class type (i.e., the statement \fe{B b = (B) ib;}). Finally, \fe{accessedMethodOwner} is the class that contains the called \fe{method} (\fe{B} in the example).

The first activity node (after the start node) creates a method declaration in the interface (\fe{methodDecl}). This new method declaration is declared as public (attribute assignment \fe{visibility := PUBLIC}) and abstract (attribute assignment \fe{abstract := true}). The declaration receives the same name as the formerly called method (attribute assignment \fe{name := method.name}, \fe{m3} in the example). The new method declaration is added to the methods of the \fe{interface} by creating a \fe{method} link between \fe{interface} and \fe{methodDecl}. The target accessed by the \fe{call} is changed by deleting the link between \fe{call} and \fe{method} and recreating it between \fe{call} and \fe{methodDecl}. The return type of the method is set by creating a new object \fe{typeAccessNew} of the type \fe{DeclarationTypeAccess} and connecting it to \fe{methodDecl}. It points to the same \fe{GASTType} as the old declaration type access of the \fe{method}.

The next node is an activity call node. It calls the story diagram \emph{Copy parameters} which is described in detail in Section~\ref{sec:SDCopyParameters}. This story diagram is responsible for copying all the parameters of the formerly called \fe{method} to the newly created declaration \fe{methodDecl}.

The following activity node contains only the two bound, mandatory object variables \fe{castStmt} and \fe{call}. It's responsibility is to try and match the link \fe{accesses} between those object variables. If the link exists that means that the cast and the call are part of the same statement. In that case the matching of the activity node is successful and the control flow continues along the transition labelled with \emph{[success]} to activity node 3a. If the matching fails, i.e., the link does not exist and the cast and the call are therefore not part of the same statement, the activity node is left via the \emph{[failure]} transition. This distinction is necessary because the effort to remove the cast statement is much greater if the cast is not done in the same statement as the call.

If the cast is in the same statement as the call, activity node 3a is executed: The \fe{castStmt} and its access to \fe{B} are deleted. If the cast is not in the same statement as the call that means that the cast has to be executed at some point before the call and the resulting downcast object has to be stored in a temporary variable. That variable is later used as a receiving object for the call. In this case, this temporary variable can be deleted along with the accesses to it from the call and the cast statements. Instead, a new variable of the interface type (\fe{IB} in the example) is created and then accessed by the call statement. In both cases, activity node 4 is executed next.

Activity node 4 is responsible for adapting all other classes that implement the now changed interface. Thus, the node is a for each node that binds a class which is connected to the \fe{interface} in each iteration. For each of those bindings, the node that is reachable via the \fe{[each time]} transition is executed (see Section \ref{sec:StoryDiagrams}). In this case that is a story diagram call of the story diagram \fe{GenerateMethodStub} which is explained in the following section.

\subsection{Story diagram: Generate method stub}

\begin{figure}[hbtp]
\centering
\includegraphics[width=0.9\linewidth]{./figures/SDGenerateMethodStub}
\caption{Story Diagram: GenerateMethodStub}
\label{fig:SDGenerateMethodStub}
\end{figure}

The story diagram \fe{GenerateMethodStub} is shown in Figure~\ref{fig:SDGenerateMethodStub}. It creates a method which implements a method \fe{methodDecl} in a given \fe{interface}. This is accomplished by three activity nodes. The first node checks if the given \fe{class} contains a \fe{method} with the same name as the given declaration \fe{methodDecl}. The check is performed by the expression \fe{'name = methodDecl.name'}. Because the object variable \fe{method} is negative (crossed-out), the matching of this activity node is considered successful if \emph{no} such method exists in the class. In that case the next activity node is executed. If a method of the name in question already exists, the execution of the first activity node fails and the story diagram terminates.

The second activity node creates a new \fe{methodStub} in the given \fe{class}. The visibility of this method is set to public and its name is set to the name of the method declaration as signified by the expression \fe{'name := methodDecl.name'}. The correct return type for the method is set by creating a \fe{newTypeAccess} from the \fe{methodStub} to the \fe{returnType} that was passed to this story diagram as a parameter.

Finally, the story diagram \fe{CopyParameters} is called in the story diagram call node. The \fe{method} and the \fe{methodStub} are passed as parameters. The called diagram then copies all parameters from the given \fe{method} to the newly created \fe{methodStub}. It is explained in the following section.

\subsection{Story diagram: Copy parameters} \label{sec:SDCopyParameters}

\begin{figure}[hbtp]
\centering
\includegraphics[width=0.9\linewidth]{./figures/SDCopyParameters}
\caption{Story Diagram: CopyParameters}
\label{fig:SDCopyParameters}
\end{figure}

The story diagram \fe{CopyParameters} (see Figure~\ref{fig:SDCopyParameters}) copies all the parameters from a \fe{sourceMethod} to a \fe{targetMethod}. Both methods are provided as parameters. The diagram consists of two activity nodes.

The first activity node is a for each node. It successively binds all formal parameters of the given \fe{sourceMethod} to the object variable \fe{param}. Each time a new parameter is bound, the second activity node is executed. There, a new formal parameter \fe{newParam} is created in the \fe{targetMethod}. Its name is set to the same name as the original parameter's by the expression \fe{'name := param.name'}. The return type is also set accordingly by binding the \fe{returnType} of \fe{param}. Then, a new access to that type is created an connected to \fe{newParam}.	

	\chapter{Related Work} 
\label{sec:RelatedWork}
In this chapter, we give an overview about scientific publications related to story diagrams.
First, we provide an extensive summary of previous work about story diagrams in Section~\ref{sec:RW_PreviousWork}, including their origins.
In Section~\ref{sec:RW_Extensions}, we report on extensions and applications of story diagrams.
Finally, we briefly describe related and similar concepts in the literature in Section~\ref{sec:RW_RelatedWork}.

\section{Origins and Previous Work on Story Diagrams}
\label{sec:RW_PreviousWork}

Story diagrams have first been described by Fischer et al. \cite{FNTZ00} and Jahnke and Z\"{u}ndorf \cite{JZ98} in 1998.
The foundations of story diagrams lie in the programmed graph rewriting systems PROGRES \cite{SWZ95} which has been developed at the University of Aachen since 1989.
Story diagrams (or story flow diagrams as they were called in early publications) adapt and enhance the PROGRES approach to a UML-like notation and an object-oriented data model \cite{JZ98}.
They have an easily comprehensible graphical syntax and well-defined semantics.
Z\"{u}ndorf \cite{Zun01} describes the syntax and semantics of story diagrams in detail.
A graph grammar that formally describes the syntax of the control flow of story diagrams was defined by Klein \cite{Kle99}.

Story diagrams are embedded in a rigorous and systematic software development method called \emph{story-driven modeling} (SDM) \cite{Zun01,DGZ04}.
While existing approaches like UML focus on the specification of the static structure of software, SDM combines, amongst others, UML class diagrams and story diagrams to allow completely specifying the structure and behavior of software systems.
Furthermore, SDM describes how such a software specification can be derived from requirements.
First, each use-case in the requirements is refined by a set of sample scenarios defined by so-called \emph{story boards}.
A story board is a sequence of single snap shots of graph-like object structures, describing changes in these object structures.
Next, the static class structure of the system is derived from the story boards and further refined.
Given the sample scenarios, the general dynamic behavior of the system is then defined using story diagrams.
Finally, the implementation of the software system can be automatically generated from these formal models.

From the beginning, tool support for story diagrams was a main focus.
\fuj, an acronym for ``From UML to Java And Back Again''\footnote{The acronym is derived from a preceding tool called FUCABA (''From UML to C++ And Back Again'') \cite{JZ97}.}, was the first tool which implemented the concept of story diagrams.
In December 1997, the project started at the University of Paderborn.
A first prototype was implemented in the course of a master's thesis \cite{FNT98}.
As story diagrams specify the behavior of software, the execution of story diagrams is an important requirement.
For instance, Z\"{u}ndorf, Sch\"{u}rr and Winter \cite{ZSW99} describe how story diagrams can be compiled into Java code.
This code generation approach was also integrated into \fuj.

A first public tool demonstration of \fuj was presented at the ICSE 2000 \cite{NNZ00}, showing advanced class and story diagram modeling facilities as well as graphical debugging and simulation.

In the following, story diagrams and \fuj have been modified and enhanced.
Originally, story diagrams used expressions of the target programming language to define constraints, return values etc.,
i.e.\ if a story diagram was to be compiled into Java code, Java expressions had to be used.
St\"{o}lzel, Zschaler and Geiger \cite{SZG07} integrated OCL into story diagrams, making them more platform-independent.
They connected \fuj to the Dresden OCL toolkit \cite{DresdenOCL}, allowing a code generation for story diagrams including the OCL constraints.

To improve flexibility for the execution of story diagrams, Giese, Hildebrand and Seibel \cite{GHS09} present an interpreter for story diagrams.
In contrast to executing generated Java code, with this approach generated story diagrams can be executed immediately.
This allows, for instance, to create higher-order transformations where story diagrams are created by other story diagrams and can immediately be executed.
As interpreting in general is slower than compiling, the authors implemented a new dynamic matching policy for their interpreter.

\todoall{Rewrite the following paragraphs with respect to the concepts presented in v0.2.}

Tichy, Meyer and Giese \cite{TMG06} identified some semantic issues in story diagrams.
First, when creating more than one element in a story pattern, the order of creation is undefined.
In general, this is no problem; however, in certain failure situations and when creating links in ordered associations, this may lead to non-deterministic behavior.
However, defining a creation order would contradict the declarative nature of story patterns.
Thus, we decided not to include an explicit creation order.
However, for the failure case, an exception transition can be defined where it can be explicitly modeled how to deal with the failure.
To define a link order in an ordered association, \fe{LinkConstraint}s can be used (see Section~\ref{cls:modeling::patterns::LinkConstraint} on Page~\pageref{cls:modeling::patterns::LinkConstraint}).
(\fe{LinkConstraint}s will be described in detail in later versions of this document.)

Second, when having a link between two set variables \fe{setA} and \fe{setB}, the intuitive semantics would be to have every set element in \fe{setA} connected to every element in  \fe{setB}.
However, this is neither supported by the tools nor allowed by the formal semantics described by Z\"{u}ndorf \cite{Zun01}.
We deal with sets in later versions of this document.

Third, consider there is a class with two qualified associations (to other classes) that have each other as a qualifiers.
When creating one link for each of the two qualified association in one story pattern, the first association that is created is qualified by the \fe{null} value although it could be qualified using the correct object (considering this is already bound).
Again, we deal with this issue in later versions of this document.

Forth, the set of possible bindings that match in a for-each activity may be extended by this very for-each activity, i.e., the activity changes something that makes new elements match for the for-each condition. In the original work on story patterns, it was not clear how this should be handled.
Thus, we define that we use a \emph{fresh matches} semantics for for-each activities (in contrast to a \emph{pre-select} semantics) in Section~\ref{sec:DecisionNodesEtc}.

Fifth, as creations may fail, e.g., due to resource constraints, the authors propose that a story diagram should be able to react to the result of a creation.
As mentioned before, an exception transition can be used to deal with such failures.

The control flow of story diagrams is modeled explicitly.
However, in certain situations, it is useful to only implicitly define the execution order, as it may significantly improve the comprehensibility of a story diagram.
Thus, Meyers and Van Gorp \cite{MG08} propose to add a new language construct for the non-deterministic selection of a execution order.

In~\cite{Sta08}, Stallmann presents an extension of story patterns which is called \emph{enhanced story pattern}. They extend story patterns by so-called \emph{insets}. Insets carry a qualifier which applies to all object and link variables in the inset. That allows to mark sub-graphs as negative, to specify \emph{and} and \emph{or} conditions on subgraphs and to qualify a subgraph by $\forall$. We will adopt these ideas in future versions of this document.

Becker et al. present means for structuring complex transformations into several independent story diagrams which can be called in a well-defined manner \cite{BvDHR11}.
They propose inventing explicit call activities which invoke other story diagrams and also support polymorphic dispatching.
Polymorphic dispatching can also be used for the aforementioned case of non-deterministic execution order.
Calls are described in Section~\ref{sec:Calls}.
We will give details on the polymorphic dispatching mechanism in story diagrams in later versions of this document.

Until 2010, different branches of story diagrams and of \fuj were developed, leading to severe difficulties when exchanging data due to incompatibilities.
In an effort to again unify the different branches, a task force was started in 2010.
A first result of this joint effort of the SDM community was a new unified and consolidated meta-model for story diagrams based on EMF \cite{HRvD+11}.
This new meta-model is the foundation for future projects; this technical report is also based on this meta-model.
One extension is the support for explicitly modeling expressions.
However, this is not described in detail here.



%motivation for the acronym Fujaba, \textit{F}rom \textit{U}ML to \textit{J}ava \textit{a}nd \textit{b}ack \textit{a}gain \cite{JZ97}

%story-driven modelling and story boards (roots of Fujaba) \cite{FNT98,FNTZ00,JZ98,ZSW99,DGZ04}

%story diagrams \cite{FNTZ00,Zun01}

%SD graph grammar for the construction of valid SDs (besides others) \cite{Kle99}

%Fujaba tool demo \cite{NNZ00}


%semantic issues \cite{TMG06}

%new meta-model \cite{HRvD+11}



\section{Applications and Extensions of Story Diagrams}
\label{sec:RW_Extensions}

In the area of reengineering, Niere et al.\ \cite{NSW+02} propose to specify design patterns with a graphical DSL which has strong relations to story patterns. In order to detect the specified patterns in source code, these DSL patterns are translated into story diagrams which are then executed through code generation. This approach has first been implemented in \fuj and later in the Reclipse Tool Suite \cite{DMT10}. In follow-up work by Fockel \cite{Foc10}, the generated story diagrams are no longer transformed into executable code but are interpreted to allow for easier debugging of the pattern specifications.

Giese and Klein extend story patterns to so called Story Decision Diagrams (SDDs) that allow to express complex safety properties \cite{GK06a}.
Basically, they require story patterns of SDDs to be non-modifying (i.e., no \create or \destroy elements) and add features of logics such as quantification, implication, and negation.
After the evaluation of such a property, a regular story pattern may be specified which describes a change operation that should be executed.

Giese and Klein also present Timed Story Scenario Diagrams (TSSDs), which are used to specify structural and temporal properties of systems in an integrated way \cite{KG07a}.

Tichy et al. \cite{THH+08} describe how story diagrams can be used to describe reconfigurations of component-based architectures, as, for instance, in MechatronicUML \cite{BBD+12}.
A transformation language called Component Story Diagrams is used to specify reconfiguration steps.
Component Story Diagrams use the concrete syntax of components for specifying the reconfiguration operations.
This language is transformed to story diagrams that can be executed to perform the actual reconfigurations.

Z{\"u}ndorf~\cite{Zue09} proposes a framework for computing the state-space of a specification in terms of story diagrams and an initial instance model. 
It has been used for model checking the leader election protocol and for a case study presented in~\cite{HSJZ10}.

Meyer \cite{Mey09} adds a few specialized constructs to story diagrams thereby extending them to transformation diagrams. Some of these extensions, such as multiple out parameters, have been integrated into the story diagrams presented in this report (see Section~\ref{sec:activities}). Similar to our complete example (Chapter~\ref{sec:Example}), Meyer uses the transformation diagrams to specify refactorings of object-oriented source code. To verify that certain properties of the code (e.g.\ variable accesses) are preserved by the refactorings, he extends Schilling's approach \cite{Sch06} to proving inductive invariants.

In~\cite{HH11b}, Heinzemann and Henkler extend story diagrams to timed story diagrams. 
Timed story diagrams are based on timed graph transformation systems~\cite{EHH+11} that extend graph transformation systems by clocks as known from timed automata~\cite{AD94}. 
They are used to model time-dependent reconfigurations of an instance model. 
In~\cite{EHH+11}, they are used as a means to define the semantics of reconfigurations in real-time systems. 
In~\cite{HSE10}, a framework for reachability analysis on timed story diagrams has been introduced which explores the state-space defined by the timed story diagrams and an initial instance model. 
It is based on the framework introduced in~\cite{Zue09}.

%\todojr{
%\begin{itemize}
%\item Design-level debugging % \cite{GZ02,Gei02,GZ06},
%\item code generation % \cite{GSR05,GBD07},
%\item reverse engineering % \cite{NSW+02,BGS+Z08}, MATE (reverse engineering) \cite{SKS+07,ST08},
%\item OCL in Fujaba % \cite{SZG07},
%\item other applications % \cite{KNNZ00,GZ10}
%\end{itemize}
%}

\section{Work Related to Story Diagrams}
\label{sec:RW_RelatedWork}

Model transformation has become an important research topic during the last years.
Several concepts and tools with different scopes and applications have been proposed.

Several model transformation approaches exist which are similar to story diagrams.

Here, we focus on those solutions that have a reasonable documentation available.
For a more comprehensive overview of transformation approaches see, for example, \cite{Czarnecki06}.
Current transformation tools can, for instance, be found in \cite{TTC2010}.

\subsection{Endogenous, In-Place Model Transformations}

\emph{Henshin}~\cite{henshin2} is a model transformation language for in-place transformations of EMF-based models.
It uses pattern-based rewrite rules (called ``transformation rules'') and control-flow-based operational semantics (called ``transformation units'') on top of it.
Transformation units can also be called by other transformation units, also including parameters.
%Henshin, however, does not provide support for polymorphic dispatching.

\emph{MOLA}~\cite{mola} is an in-place model transformation language with a graphical syntax similar to story diagrams.
Transformation rules may consist of multiple matching and modification patterns and the control flow inside a transformation rule can be specified with a focus on the loop construct.
Furthermore, it also allows calling other transformations rules. %, but does not support polymorphic dispatching.

\emph{Groove}~\cite{Ren04a} is a graph transformation tool with a focus on analyzing graph transformation systems.
Its rules consist of single rewrite patterns.
For instance, given a rule set and a start graph, Groove can explore the graph state space and use this for model checking.
It also features so called ``control programs'' which allow the user to restrict which rules can be applied and in which order. It provides model checking of LTL properties~\cite{Ren08} and CTL properties~\cite{KR06}.

\emph{VIATRA}~\cite{viatra}, a textual language, uses abstract state machines to specify the control flow and graph transformation rules for elementary model manipulations.
It also addresses modularization by reusable patterns that are called from the graph transformation rules. 

%Although most of the story-diagram-like transformation languages include means for specifying control flow (including calling other transformation units), none of them supports polymorphic dispatching.
%(Note that in most cases polymorphic dispatching can be emulated using other means, but doing so would result in a more complex and less maintainable rule set.)

%However, looking at other transformation language types, there are some approaches that support polymorphic dispatching. 
%For instance, \emph{Xpand}~\cite{xpand}, a model-to-text transformation language based on templates, uses ``polymorphic template invocation'' where the most %specialized template available is used.
%However, it only supports single dispatch, i.e., only one parameter is used to determine the used template.   

\subsection{Exogenous, Inter-Model Transformations}

In general, Story Diagrams can also be used to specify inter-model transformations.
In this case, a story diagram would contain elements from both the source and the target model.
If necessary, a trace model could also be created.
In comparison to dedicated inter-model transformation languages, story diagrams may be more tedious to use in this application scenario.
However, when a transformation requires extensive pre-computations or complex distinction of cases, story diagrams are a reasonable alternative.

\emph{QVT Operational}~\cite{QVT} is a operational model transformation language designed for writing unidirectional transformations which is part of the OMG QVT standard. %that also allows polymorphic dispatching by its \fe{disjuncts} keyword.
%A QVT-O mapping operation can declare that a call to it should be dispatched to other mappings.
%In this case, the invocation of that mapping operation results in the execution of the first mapping whose signature fits the concrete parameters and whose \fe{when} clause evaluates to \fe{true}.
%This is a more powerful construct than our solution, as it not only allows dispatching based on the actual type, but arbitrary constraints.
%However, there must be a base rule where all dispatching possibilities are listed;
%in our solution, all signature-compatible transformations with the same name are automatically used in dispatching, allowing a better modularization as well as an easier extension of the rule set.      
However, QVT-O is a textual transformation language which may not be well-suited in many cases~\cite{Moo09}.

In declarative inter-model transformation languages like \emph{Triple Graph Grammars} (TGGs) \cite{Sch94}, the control flow cannot be defined explicitly.
Instead, the order of the rule application is implicitly defined by preconditions of the transformation rules.
However, when more than one rule has a fitting precondition, the rule to be applied is selected non-deterministically, dependent on the concrete transformation tool implementation, or by a given rule priority.
This can make the comprehension of a TGG rule set difficult.
%Klar et al. \cite{Klar07} proposed a rule generalization concept with a precedence for the most refined rules, a solution similar to polymorphic dispatching.

In \emph{QVT Relations}~\cite{QVT}, which is similar to TGGs, control flow may also be explicitly specified by using \fe{where} clauses.

The \emph{Atlas Transformation Language} (ATL)~\cite{ATL} is a hybrid inter-model transformation language, integrating declarative and operational aspects.
It is similar to QVT, but only has a textual representation of the transformation rules.

%\todojr{
%Further related work:
%\begin{itemize}
%\item AGG
%\item Evtl. Loesungen zu M2M Transformation Aufgabe aus dem Tool Transformation Contest 2010 fuer weiteres Related Work anschauen
%\item GReTL % \cite{HE11}
%\end{itemize}
%}
	
	\chapter{Conclusions and Future Work (Markus)}

\section{Summary}

\section{Future Work on report}
\begin{itemize}
  \item Description of advanced concepts
  \item Formalization
  \item Verification
\end{itemize}

\section*{Acknowledgement}

Reference former contributors and possibly their theses.
	
	%\setstretch{1}
        
% -------------------------------------------------------------
% Literaturverzeichnis 
% -------------------------------------------------------------
	\addtocontents{toc}{}         % Zum Inhaltsverzeichnis muss eine Leerzeile zugef"ugt werden, weil
                        	      % sonst das Literaturverzeichnis falsch im Inhaltsverzeichnis erscheint.
	\bibliographystyle{alpha}

	\bibliography{SdmPaper}

% ------------------------------------------------------------
% Appendix
% ------------------------------------------------------------

\appendix

	\chapter{User Guide}\label{sec:UserGuide}
	
This section should contain a user guide for the available SDM tools.


\section{Installation}


\subsection{Installation Using the Eclipse Update Site -- Users}


\subsection{Getting the Source Code From Repository -- Developers}


\section{Getting Started -- User Interface}

	\chapter{Execution of Story Diagrams (Stephan)}

In general, there are two possibilities to execute models: Executing them directly using an interpreter \cite{GHS09} or generating GPL code, which is either compiled or interpreted. 
\fuj can generate Java or C code from story diagrams and their accompanying class diagrams. 
Here, a story diagram describes the behavior of a single method. 
Therefore, this method and its containing class must be defined first. 
Interpreting a story diagram does not impose this restriction.
In the following sections, we describe the structure and operation principles of an interpreter for story diagrams.


\section{Interpreting Story Diagrams}
\label{sec:InterpretingStoryDiagrams}

\subsection{Interpreter Architecture}

\begin{figure}[htb]
  \centering
  \includegraphics[width=1.0\columnwidth]{./figures/interpreter_packages.pdf}
  \caption{Overview of the Packages of the Interpreter}
  \label{fig:interpreter_packages}
\end{figure}

Figure~\ref{fig:interpreter_packages} shows the package structure of the SDM interpreter.
Currently, there are multiple story diagram metamodels in use that must all be supported by the interpreter. 
Therefore, the interpreter is divided into a metamodel-independent core (\emph{de.mdelab.sdm.interpreter.core}) and multiple metamodel-dependent extensions (\emph{org.storydriven.modeling.interpreter} and \emph{de.mdelab.sdm.interpreter.sde}). 
This separation of metamodel-dependent and independent parts allows for easier maintenance of the interpreter. 
The classes of the core package define a quite extensive list of generic type parameters (not shown in the subsequent class diagrams), e.g., for activity nodes, classifiers, or features. 
Subclasses in the metamodel-dependent packages replace these generic types with the concrete types defined in the respective metamodel.

Furthermore, those parts of the interpreter that depend on Eclipse are also separated (\emph{*.eclipse} packages). 
This allows to use the interpreter in stand-alone applications without Eclipse. 
In addition, the interpreters for expression languages like OCL are also separated (\emph{de.mdelab.sdm.interpreter.ocl}). 
The SDM interpreter provides an extension mechanism to add interpreters for other expression languages.

\subsubsection{SDM Interpreter}
\label{sec:sdm_interpreter}

\begin{figure}[htb]
  \centering
  \includegraphics[width=1.0\columnwidth]{./figures/interpreter_core.pdf}
  \caption{Main classes of the interpreter core}
  \label{fig:sdm_interpreter}
\end{figure}

Figure~\ref{fig:sdm_interpreter} shows the main classes of the interpreter core. 
\emph{SDMInterpreter} is the abstract superclass of all SDM interpreters. 
It is responsible for the execution of a whole story diagram. 
\emph{StoryDrivenInterpreter} and \emph{StoryDrivenEclipseInterpreter} inherit from it to implement metamodel specific functionality. 
Here, this is mostly limited to defining the concrete types to use in place of the generic types of the superclass and defining constructors with fewer parameters. 
The \emph{SDMInterpreter} provides the \emph{executeActivity()} method to execute a story diagram.

A \emph{VariableScope} is a collection of \emph{Variable}s that are valid in a specific scope.
A \emph{Variable} is a triple of the name, the classifier, and the value of the variable.
The \emph{SDMInterpreter} maintains multiple \emph{VariableScope}s, one for each activity node.
Currently, the whole story diagram forms a single scope.
All variables created in a story diagram are also valid in the whole story diagram. 
Therefore, the same scope is used for all activity nodes.
Only, when other story diagrams are called, a new variable scope is created. 
However, if additional elements are added to story diagrams, e.g., fork and join nodes, or the semantics of existing elements is changed, it is easily possible to support separate scopes for each activity node.

The \emph{ExpressionInterpreterManager} is responsible for managing the interpreters for expression languages and delegating the evaluation of an expression to the appropriate \emph{ExpressionInterpreter}. 
The \emph{evaluateExpression()} method is provided for that purpose. 
\emph{ExpressionInterpreter}s have to be registered at the \emph{ExpressionInterpreterManager} via the \emph{registerExpressionInterpreter()} method before expressions of that language can be handled. 
The \emph{EclipseExpressionInterpreterManager} performs this registration automatically. 
The plug-in \emph{de.mdelab.sdm.interpreter.eclipse} defines an extension point for \emph{ExpressionInterpreter}s. 
All interpreters extending this extension point are registered automatically by the \emph{EclipseExpressionInterpreterManager}. 
If the interpreter is not used within Eclipse, \emph{ExpressionInterpreter}s have to be registered explicitly before executing a story diagram.

The abstract class \emph{ExpressionInterpreter} only defines the \emph{evaluateExpression()} method that must be implemented by subclasses such as the \emph{OCLExpressionInterpreter} and the \emph{CallsInterpreter}. The methods performs the execution of the expression, which may have side effects, and has to return a \emph{Variable} with the return type and return value of the expression. In this methods, the current \emph{VariableScope} can also be access and modified.

The interpreter often needs to access specific properties of story diagram elements, e.g., the name of elements or incoming and outgoing edges of activity nodes.
While the interpreter core is metamodel-independent, it cannot access these properties directly but needs a facade for that purpose.
The \emph{MetamodelFacadeFactory} provides access to these facades. 
There are several interfaces for common kinds of story diagram elements (e.g., story nodes, junction nodes, object variables or link variables), which have to be implemented for specific story diagram metamodels.

A \emph{StoryPatternMatcher} is responsible for the execution of a single story pattern. 
This abstract superclass defines the methods \emph{findNextMatch()} to search for the next match of a story pattern and \emph{applyMatch()} to execute the graph transformation on the last match. 
The class \emph{StoryPatternMatcher} does not implement a particular matching strategy. 
This is done by \emph{PatternPartBasedPatternMatcher}. 
This pattern matching strategy is explained in more detail in Section~\ref{sec:story_pattern_matcher}. 


\subsubsection{Story Pattern Matcher}
\label{sec:story_pattern_matcher}

\begin{figure}[htb]
  \centering
  \includegraphics[width=1.0\columnwidth]{./figures/interpreter_storyPatternMatcher.pdf}
  \caption{Main classes of the story pattern matcher}
  \label{fig:interpreter_storyPatternMatcher}
\end{figure}

Figure~\ref{fig:interpreter_storyPatternMatcher} shows the classes of the story pattern matcher. 
Currently, only one pattern matching approach is implemented. 
The \emph{PatternPartBasedPatternMatcher} splits the story pattern into multiple \emph{PatternParts}. 
What exactly constitutes a \emph{PatternPart} is not specified in the interpreter core. 
This has to be implemented in the metamodel specific subclasses. 
Currently, the \emph{StoryDrivenPatternMatcher} enforces the following semantics: A pattern part consists either of a single variable that has no incoming or outgoing links (\emph{VariableOnlyPatternPart}), or of a single link and its adjacent object variables (\emph{StoryDrivenLinkVariablePatternPart}, \emph{StoryDrivenContainmentRelationPatternPart}, and \emph{StoryDrivenPathPatternPart} depending on the kind of link). 
This implies that a variable can be contained in more than one pattern parts. 
To implement subpatterns or complex negative application conditions, this semantics has to be modified. 
But this remains transparent to the basic \emph{PatternPartBasedPatternMatcher}.

\emph{PatternPart} defines several abstract methods:

\begin{enumerate}
	\item \emph{getMatchType()} returns whether matching the pattern parts is mandatory or optional, or whether the pattern part is a negative application condition (cf. Section~\ref{sec:StoryPatterns:binding:semantics}). 
	
	\item \emph{check()} checks whether the link exists in the instance graph, which requires that all variables of the pattern part are already bound to an instance object. 
	If this is not the case, \emph{check()} returns \emph{UNKNOWN}. 

	\item \emph{calculateMatchingCost()} provides an estimate of the cost to match a variable using the link of the pattern part. 
	This estimate can be based, e.g., on the number of elements contained in the link.
	If it is currently not possible to match this pattern part (e.g., because all variables of the pattern part are still unbound), \emph{-1} is returned.

	\item \emph{match()} implements the pattern matching for this kind of pattern part. 
	It is called after \emph{calculateMatchingCost()}. 
	To find a match, at least one variable of the pattern part has to be bound and one unbound. 
	Then, \emph{match()} tries to find matches for all unbound variables. 
	This part of the pattern matching algorithm is highly implementation specific. 
	It is not only different for different metamodels. 
	It also has to be implemented differently for different kinds of link variables.
	For example, matching an object via an ordinary \emph{LinkVariable} has to be done differently than matching via a \emph{Path} or a \emph{ContainmentRelation}.
	However, this also allows to exploit certain features of the metamodel to improve execution performance. 
	For example, \emph{StoryDrivenContainmentRelationPatternPart} uses EMF's \emph{eContainer()} method to navigate containment links in the opposite direction.	
	
	\item \emph{createLinks()} and \emph{createObjects} create those elements of the pattern part, that are marked with \create.
	
	\item \emph{destroyLinks()} and \emph{destroyObjects()} destroy links and objects. 
	In contrast to the creation of elements, these steps are separated to ensure an orderly deletion of story pattern variables in the \emph{VariableScope}.
	\footnote{Background: There are two ways to execute story patterns with deleted elements: Destroy all links first and then all objects, or the other way round. 
	EMF also supports unidirectional references. 
	Therefore, deleting objects as implemented in \emph{EcoreUtil.delete()} is done by going from the deleted object to the root of the containment hierarchy (usually the \emph{Resource} or \emph{ResourceSet}) and searching for cross-references to the deleted object. 
	If the links are deleted first when the story pattern is executed, the destroyed object may be removed from its containment hierarchy (if a destroyed link represents this containment). 
	After that, existing cross-references to the destroyed object that are not represented by a link in the story pattern (remember that story patterns have SPO semantics) cannot be found and deleted. 
	For this reason, the interpreter first deletes all objects and then all links.}
	

\end{enumerate}


The \emph{MatchingStrategy} determines the order in which pattern parts are matched. 
The \emph{DefaultMatchingStrategy} matches pattern parts in the order of their matching cost estimates, i.e., \emph{getNextPatternPartForMatching()} returns that pattern part with the lowest cost estimate. 

There are also two additional pattern matching strategies: \emph{DefaultMatchingStrategyWithLog} and \emph{LogReproducingMatchingStrategy}. 
These are required for \emph{for-each} story nodes. 

\subsubsection{Notification Mechanism}
\label{sec:notification_mechanism}

\begin{figure}
\includegraphics[width=1.0\columnwidth]{figures/interpreter_metamodel_facades.pdf} 
\caption{Relevant classes of the interpreter's notification mechanism}
\end{figure}

The interpreter and all its components provide a notification mechanism to inform clients of all important steps during the execution of a story diagram, which is an implementation of the observer design pattern.

All active components of the interpreter (e.g., \emph{SDMInterpreter} or \emph{StoryPatternMatcher}), extend the \emph{Notifier} superclass.
\emph{Notifier} defines a reference to a \emph{NotificationEmitter}. 
This class provides an operation for each type of notification defined in \emph{NotificationTypeEnum} that creates an \emph{InterpreterNotification} and forwards it to all registered \emph{NotificationReceivers} by calling their \emph{notifyChanged()} operations. 
Clients can add their own \emph{NotificationReceiver}s to the \emph{NotificationEmitter}'s list of receivers.

By default, each \emph{Notifier} uses the default implementation in \emph{NotificationEmitter}. 
However, it is also possible to create \emph{Notifiers} with custom implementations of \emph{NotificationEmitter} to directly process notifications there or process notifications asynchronously, for example.


\subsection{Interpreting Story Diagrams}

\begin{figure}[htb]
\center
\includegraphics[scale=0.7]{figures/SDInterpreterExecution.pdf} 
\caption{Execution Scheme of the \emph{SDMInterpreter}}
\label{fig:sdmInterpreter_execution_scheme}
\end{figure}


The overall interpretation of a story diagram is a simple graph traversal algorithm.
The interpretation starts at the story diagram's \emph{StartNode} and traverses the story diagram until it reaches a \emph{StopNode}. 
All activity nodes are executed by specialized methods, which return the next activity node to execute afterwards. The activity diagram in Figure~\ref{fig:sdmInterpreter_execution_scheme} shows the overall execution scheme of the interpreter.

The interpreter is started with the \emph{executeActivity()} method.
This method creates the root \emph{VariableScope} and a \emph{Variable} for each parameter of the story diagram.
Then, the \emph{StartNode} of the story diagram is obtained and executed.

In general, the execution of activity nodes works as follows: 
First, the kind of the activity node is determined. Then, it is executed by the appropriate execution method.
\emph{StoryNode}s, \emph{JunctionNode}s, \emph{StopNode}s, and \emph{StatementNode}s require special handling.
All other kinds of activity nodes are simply skipped.
After execution of a node, the next node to execute is returned by the handling methods.
This process is repeated until a \emph{StopNode} is reached, which has no subsequent nodes.
Then, the loop terminates. The return value expressions of all outgoing parameters are evaluated and put into a map, which is returned by \emph{executeActivity()}.
This map maps the parameter names to their values.\footnote{For backward compatibility, all variables of the story diagram are currently returned, not only parameters.}

A non-for-each \emph{StoryNode} is executed using the \emph{StoryPatternMatcher} (cf. Section~\ref{sec:story_pattern_matcher}) with the \emph{DefaultMatchingStrategy}. 
It searches for a match of the story pattern and applies the graph transformation rule if a match was found. 
For for-each nodes, the process is more complex. 
If the story pattern is executed for the first time, a new \emph{StoryPatternMatcher} is created and stored in a local map for this \emph{StoryNode}. 
The pattern matcher is executed with the \emph{DefaultMatchingStrategyWithLog}, which keeps a log of the order in which elements were matched. 
If a match was found, the next activity node is returned that is connected via a for-each edge. 
The interpreter executes that node and eventually the control flow returns to the for-each node.
Now, the existing \emph{StoryPatternMatcher} is reused so that it continues pattern matching where it left of. 
This time, however, the pattern matcher uses the \emph{LogReproducingMatchingStrategy}. 
This ensures, that all elements are matched in the same order as the first time. 
For-each nodes are executed with the \emph{fresh match} semantics. 
After a match was found, the story pattern's side effects are executed immediately. 
Then, the next match is sought. 
Side-effects may influence subsequent matches, they may create new matches or eliminate existing ones. 
They may also influence they matching order if they change the number of elements in references, which changes the cost estimates (for details, see Section~\ref{sec:interpreting_story_patterns}). 
If the \emph{DefaultMatchingStrategy} would be used in each iteration of the for-each node, it may choose a different matching order in subsequent iterations. 
Due to the way how previous matches are managed, this may cause the pattern matcher to return a match multiple times or skip valid matches.
Therefore, the \emph{DefaultMatchingStrategyWithLog} is used in the first iteration of a for-each story node to log the matching order, and the \emph{LogReproducingMatchingStrategy} is used in all subsequent iterations, which matches elements in exactly the same order.

After the last iteration, when the pattern matcher did not find any more matches, the stored mapping between the \emph{StoryNode} and the pattern matcher is discarded. 
If the story diagram's control flow returns to the for-each node again, the pattern matching process starts anew.

In addition to the \emph{fresh match} semantics, it is also possible to add other execution semantics for for-each nodes, e.g., \emph{pre-select}, which searches for all matches before executing side-effects.


\subsection{Interpreting Story Patterns}
\label{sec:interpreting_story_patterns}

The execution of a single story pattern comprises three steps:
Initialization of the pattern matcher and analysis of the story pattern (Section~\ref{sec:spm_initialization}),
pattern matching (Section~\ref{sec:spm_pattern_matching}),
and pattern application (Section~\ref{sec:spm_pattern_application}).
These steps are executed in the constructor of the pattern matcher, the \emph{findNextMatch()}, and the \emph{applyMatch()} operations respectively.
\emph{findNextMatch()} can be called successively to return all matches for a story pattern one-by-one.
These operations are separated, to allow for additional operations between these phases by the user of the pattern matcher.


\subsubsection{Initialization and Pattern Analysis}
\label{sec:spm_initialization}

A \emph{StoryPatternMatcher} is instantiated for a specific story pattern. 
Therefore, the \emph{StoryPatternMatcher}'s constructor already requires the story pattern as a parameter.
In the constructor, the matcher's internal data structures are set up. 
These comprise lists of the bound and unbound pattern variables\footnote{Subsequently, we use the term \emph{pattern variable} to refer to object variables or primitive variables in a story pattern in contrast to \emph{Variable}s to refer to \emph{Variable} objects used internally by the pattern matcher.}, checked and unchecked \emph{PatternPart}s, bound instance objects, the matching history, and the stack of match transactions.
The matching history is a mapping between pattern variables and lists of instance objects, that were previously bound to that pattern variable.
The match transaction stack is a stack that contains stack elements for each relevant action of the matcher.
Each time, a pattern part is matched or checked, or a pattern variable is bound to an object, a match transaction is executed and pushed onto the stack.
A transaction usually involves a manipulation of the internal data structures of the matcher, e.g., moving an element from the list of unbound to that of bound pattern variables.
When the matcher has to step back, these match transactions are popped from the stack and rolled back.

After initializing these data structures, the story pattern is divided into pattern parts.
Currently, each unconnected pattern variable and each link variable with its connected pattern variables form a pattern part.
This implies that a single pattern variable may belong to multiple pattern parts.
The matching process works on the granularity of these pattern parts.
This is the current default semantics specific for the two introduced metamodels.
This semantics can be different for other metamodels and it can be changed to support, e.g., complex negative application conditions.


\subsubsection{Pattern Matching}
\label{sec:spm_pattern_matching}

\begin{figure}
\begin{verbatim}
create child variable scope of current variable scope;

bind bound objects;

check unchecked pattern parts;

boolean match = true;

do {
  while (nextPatternPart = 
      matchingStrategy.getNextPatternPart() != null) {
    
    commit matchPatternPart transaction;
    match = nextPatternPart.match();
    
    if (not match) {
      roll back last two matchPatternPart transactions;
      
      if (matchingStack is empty)
        break;
    }
  }
    
  if (match) {
    match = checkStoryPatternConstraints();
    if (not match)
      roll back last matchPatternPart transaction;
  }
} while (matchingStack is not empty and not match)

if (match) {
	merge child variable scope into its parent scope;
}
\end{verbatim}
\caption{Overall pattern matching algorithm}
\label{listing:pattern_matcher_execution}
\end{figure}

\emph{findNextMatch()} is responsible for searching for the next valid match of the story pattern in the instance model.
The operation returns a boolean value indicating whether a match was found or not.
If a match was found, the pattern matcher's \emph{VariableScope} is manipulated accordingly.
If no match was found, the \emph{VariableScope} is left untouched.
Fig.~\ref{listing:pattern_matcher_execution} shows the overall scheme of this method.

First, a new variable scope is created, which is a child of the current variable scope.
During pattern matching, this child variable scope is modified but its parent is left untouched.
Next, all pattern variables that are marked as bound are bound to the appropriate variables in the \emph{VariableScope}.
Pattern variables with binding expressions are also handled here.
After that, all unchecked pattern parts are checked.
After binding all pattern variables, some pattern parts may contain only bound objects.
These pattern parts can be checked already at this point.
If a check fails, there can be no match for the story pattern and the pattern matcher terminates.
Otherwise, the actual pattern matching algorithm starts.

In two nested loops, the matching strategy returns the next pattern part to use for matching.
The matching strategy may use arbitrary heuristics to choose a pattern part from the list of unchecked pattern parts.
The default strategy returns that pattern part with the lowest cost estimate.
This pattern part must contain at least one bound and one unbound object and it must be possible to navigate from the bound objects to the unbound objects.
The \emph{calculateMatchingCost()} operation used in both metamodel specific implementations checks this.
A transaction for matching the current pattern part is pushed on the stack and the pattern part's \emph{match()} operation is called.
This operation is specific to the actual type of the link of the pattern.
In case of ordinary \emph{LinkVariables}, \emph{match()} simply follows the link from the bound instance object and tries to bind the unbound pattern variable to an instance object of the object graph.

If matching the pattern part was successful, i.e. \emph{match()} returned true, the loop continues with the next pattern part.
When all pattern parts have been matched, the matching strategy returns null instead of a pattern part.
Now, all constraints are checked that are defined for the while story pattern.
If these conditions are also satisfied, a valid match has been found.
Now, the child variable scope is merge into its parent scope to persist the match.

In case the \emph{match()} operation did not find a match, the last two transaction for matching pattern parts have to be rolled back.
Of course, this also rolls back all bindings of pattern variables that were performed in the meantime.
Here, also the second last transaction has to be rolled back because the pattern variable matched in that transaction has now shown to be an invalid match and a new match has to be found for it.
A roll back is also performed if the constraints defined on the whole story pattern are not satisfied.
If a roll back leads to an empty matching stack, the pattern matching process is terminated because no valid match exists.


\subsubsection{Pattern Application}
\label{sec:spm_pattern_application}

After finding a match for a story pattern, the story pattern can be applied, i.e. its side-effects can be executed.
This is done in \emph{applyMatch()}, which must be called be the caller of the pattern matcher explicitly.
First, all attribute assignments are evaluated and its results are stored in a list.
After that, all objects marked as \destroy are deleted and then all links.
Finally, all objects and links marked for creation are created and all attributes are assigned their new values.

It is important that all objects are deleted before deleting any links.
Object deletion is performed via the \emph{EcoreUtil.delete()} operation, which also deletes all cross-references to the deleted object.
To do so, this operation walks the containment hierarchy upwards and searches for cross-links in the whole model tree.
If the pattern matcher would delete links first, it might not be possible to walk the containment hierarchy upwards if the deleted link pointed to the container of a deleted object.
Then, it would be impossible to delete all cross-references.

Attribute assignment expressions are evaluated at the beginning to support expressions that refer to deleted elements. 
If attribute assignments would be evaluated at the end, it would no be possible to evaluate expressions with references to deleted elements because these do not exists anymore in the \emph{VariableScope}.


\section{Code Generation for Story Diagrams}
As stated in the introduction of this chapter, a second possibility of executing story diagrams besides interpretation is code generation.
When story diagrams were originally conceived, this was the first approach to executing them.
The \fuj Tool Suite \cite{KNNZ99,NNZ00} provided the possibility to generate Java code from the story diagrams that could then be executed.
This functionality was retained throughout several versions of \fuj and even extended to allow the generation of EMF-compatible code \cite{GBD07}.
With the creation of the new metamodel for story diagrams \cite{HRvD+11}, this option was discontinued in favour of the interpretation described in Section~\ref{sec:InterpretingStoryDiagrams}.

	\chapter{Technical Reference}\label{sec-reference}
	
	\input{08_TechDoc_Contents}	

% ------------------------------------------------------------
% Abbilungsverzeichnis
% ------------------------------------------------------------
	%\listoffigures


% -------------------------------------------------------------
% Tabellenverzeichnis 
% -------------------------------------------------------------
    %\listoftables
	

	% ************************************************************
	% *****                      Ende                        *****
	% ************************************************************

\end{document}
