\section{Story Patterns and Story Diagrams in a Nutshell (Jan and Dietrich)} \label{sec:Overview}

\tododt{Suggestion for an overview section}

- story diagrams are special UML activity diagrams
- developed to formalize the description of a software's behavior (UML activity diagrams usually use informal textual descriptions of the tasks to be performed)
- motivation: complete specification of a software, structure and behavior, i.e. make the software specification executable (code generation and interpretation)

- motivation: (formally) describe modifications of object structures for object-oriented software systems
- motivation: use graphically specified operations, stay in the OO world, (hopefully) more intuitive and on a higher abstraction level than other languages
- motivation: declaratively describe the operations in activity nodes, thus, reduce complexity (avoid describing how to perform the operations)

- story diagrams use graph transformations in their activity nodes (well-known formalism, exhaust the given theories for analyses)
- given a so called host graph (an object structure or model), story diagrams describe the graph's modifications by means of creating or removing objects and their interconnections
- the host graph, in our case, is a typed attributed graph, i.e. we have a graph to be modified (model) and a corresponding type graph (meta-model) describing the types and properties of the objects in our host graph






\subsection{Application scenarios (?)} \label{sec:Applications}