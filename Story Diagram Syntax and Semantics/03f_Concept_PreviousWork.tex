\section{Previous Work (Jan)}

\todomvd{This ``Short history of Fujaba'' seems odd here. Maybe we should it put at the end of the introduction?}

Story diagrams have been described first by Fischer et al. \cite{FNTZ00} and Jahnke and Z\"{u}ndorf \cite{JZ98} in 1998.
The foundations of story diagrams lay in the programmed graph rewriting systems PROGRES \cite{SWZ95} which has been developed at the University of Aachen since 1989.
Story diagrams (or story flow diagrams, as they have been called in early publications) adapt and enhance the PROGRES approach to a UML-like notation and an object-oriented data model \cite{JZ98}, using an easily comprehensible graphical syntax and well-defined semantics.
Z\"{u}ndorf \cite{Zun01} describes the syntax and semantics of early story diagrams in detail.
A graph grammar that formally describes the syntax of the control flow of story diagrams was defined by Klein \cite{Kle99}.
Story diagrams are embedded in a rigorous and systematic software development method called story driven modeling \cite{Zun01,DGZ04}.

From the beginning, there was strong tool support for story diagrams.
In December 1997 the Fujaba project started at the University of Paderborn.
A first prototype was implemented in the course of a master thesis \cite{FNT98}.
Fujaba, an acronym for ``From UML to Java And Back Again''\footnote{The acronym is derived from a preceding tool called FUCABA (''From UML to C++ And Back Again'') \cite{JZ97}.}, combines UML class diagrams, UML activity diagrams, and story diagrams to allow completely specifying the structure and behavior of software systems.
These specifications can then be executed.
For instance, Z\"{u}ndorf, Sch\"{u}rr and Winter \cite{ZSW99} describe how story diagrams can be compiled into Java code.
A first public tool demonstration of Fujaba was presented at the ICSE 2000 \cite{NNZ00}, showing advanced class and story diagram modeling facilities as well as graphical debugging and simulation.

In the following, story diagrams and Fujaba have been modified and enhanced.
Originally, story diagrams use expressions of the target programming language to define constraints, return values etc.
I.e., if a story diagram should be compiled into Java code, Java expressions must be used.
St\"{o}lzel, Zschaler and Geiger \cite{SZG07} integrated OCL into story diagrams, making them more platform-independent.
They conntected Fujaba with the Dresden OCL toolkit, allowing a code generation for story diagrams including the OCL constraints.

SD interpreter \cite{GHS09}

Tichy, Meyer and Giese \cite{TMG06} identified some semantic issues in story diagrams.


%motivation for the acronym Fujaba, \textit{F}rom \textit{U}ML to \textit{J}ava \textit{a}nd \textit{b}ack \textit{a}gain \cite{JZ97}

%story-driven modelling and story boards (roots of Fujaba) \cite{FNT98,FNTZ00,JZ98,ZSW99,DGZ04}

%story diagrams \cite{FNTZ00,Zun01}

%SD graph grammar for the construction of valid SDs (besides others) \cite{Kle99}

%Fujaba tool demo \cite{NNZ00}


%semantic issues \cite{TMG06}

new meta-model \cite{HRvD+11}
