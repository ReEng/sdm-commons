\chapter{Introduction (Jan)}
% Problem Area
To tackle the complexity of modern technical systems, the engineering of such systems is heavily based on models.
Models are considered first-class artifacts of the development.
They describe different parts of the system in development, from different viewpoints and on different abstraction levels.
For instance, models are used to describe the structure and behavior of a software system, improving the overall comprehensibility of the system.
These models can then be employed to automatically generate code, reducing the risks of implementation errors.

As models evolve during the development process and different models have to be translated into each other and kept consistent, model transformations are a crucial part such a development process.
Model transformations are used to define in which way models can be changed, e.g., to specify refactoring operations.

Furthermore, model transformations itself can be employed to precisely specify the behavior of a system at run-time.
If, for example, a system should react to environment changes by reconfiguration, these reconfigurations can be described by model transformations which define how to reconfigure the system.
They furthermore allow a formal analysis, e.g., to prove that certain properties still hold after applying a transformation.

Story diagrams~\cite{ZSW99,FNTZ00,Zun01} are a powerful visual formalism for specifying model transformation, based on the well-known concept of graph transformation systems.
They feature declarative parts to specify object patterns which are matched and altered in the source model and combine them with an imperative part to specify the control flow of the transformation execution.
The concrete syntax of story diagrams extends the concrete syntax of UML activity diagrams.

% Problem
Since their introduction in 1999, several extensions to story diagrams have been proposed.
Furthermore, some semantic issues have been identified in the original concept~\cite{TMG06}.
In addition, the main story diagram tool, the Fujaba tool suite, has undergone major redesigns the last years; these redesigns also affected the story diagram implementation.
Moreover, new approaches like the Story Diagram Interpreter have emerged.

% Approach
In this technical report, we seek to provide a complete reference of the syntax and semantics of story diagrams.
It consolidates previous publications~(\todojr{hier alle wichtigen Papiere auflisten}) in a single document.
We provide formal definitions for the abstract and the concrete syntax as well as the semantics of story diagrams.\todoall{Beschreiben wir die Semantik formal?}

%Evaluation
As an example, we show how story diagrams can be used to specify refactoring operations on structural software models like class diagrams.

% Structure
\todojr{Struktur}


\todoall{Address the issues raised in \cite{TMG06} in this report.}

\section*{Old stuff from rejected paper}
% Problem Area
In model-driven software engineering, model transformations play a central role in transforming models of higher abstraction levels into more concrete models. 
Such transformations are written in special purpose languages which offer explicit support for common transformation tasks like matching elements of the source model.
While the development of current model transformation languages focused on those model transformation specific tasks, classical issues like inheritance and structuring were neglected.
As a result, the transformations are often concise and efficient but hardly maintainable.

% Problem
Story diagrams~\cite{ZSW99,FNTZ00,Zun01} form a special model transformation language.
They feature declarative parts to specify object patterns which are matched and altered in the source model and combine them with an imperative part to specify the control flow of the transformation execution.
The concrete syntax of story diagrams extends the concrete syntax of UML activity diagrams.
A specific challenge for story diagrams is the missing support for the invocation of other story diagrams.
We require these invocations to account for aspects like the increased complexity of binding parameters and result values in a graphical language as well as dynamic dispatching of calls.

% Related Work
Some of the related transformation languages like ATL, QVT Operational, or Henshin already offer some support for structuring transformations into sub transformations.
However, no existing hybrid and graphical transformation languages addresses all of the aforementioned challenges. 

% Approach
In this paper, we present an approach to tackle the lacking feature of invoking story diagrams.
Our solution supports binding the parameter values as well as the result values of the invoked diagram.
We also introduce polymorphic dispatching in our solution, i.e., the invoked diagram is chosen at run-time based on the actual types of the bound parameter values.
Consequently, the story diagram meta-model as well as the existing interpreter for story diagrams \cite{GHS09} have to be extended to support the new concepts. 

% Validation
We show the effectiveness of our approach in a qualitative study of an existing transformation implemented in story diagrams. In the study, we estimate the potential to reduce the total amount and size of the story diagrams in our initial implementation. The results show a significant improvement in transformation size, leading to reduced complexity and thus an increase in the long-term maintainability of the transformation.

% Contribution summary
