\chapter{Introduction (Jan)}
% Problem Area
To tackle the complexity of modern technical systems, the engineering of such systems is heavily based on models.
Models are considered first-class artifacts of the development.
They describe different parts of the system in development, from different viewpoints and on different abstraction levels.
For instance, models are used to describe the structure and behavior of a software system, improving the overall comprehensibility of the system.
These models can then be employed to automatically generate code, reducing the risks of implementation errors.

As models evolve during the development process and different models have to be translated into each other and kept consistent, model transformations are a crucial part such a development process.
Model transformations are used to define in which way models can be changed, e.g., to specify refactoring operations.

Furthermore, model transformations itself can be employed to precisely specify the behavior of a system at run-time.
If, for example, a system should react to environment changes by reconfiguration, these reconfigurations can be described by model transformations which define how to reconfigure the system.
They furthermore allow a formal analysis, e.g., to prove that certain properties still hold after applying a transformation.

Story diagrams~\cite{ZSW99,FNTZ00,Zun01} are a powerful visual formalism for specifying model transformation, based on the well-known concept of graph transformation systems.
They feature declarative parts to specify object patterns which are matched and altered in the source model and combine them with an imperative part to specify the control flow of the transformation execution.
The concrete syntax of story diagrams extends the concrete syntax of UML activity diagrams.

% Problem
Since their introduction in 1998, several extensions to story diagrams have been proposed.
Furthermore, some semantic issues have been identified in the original concept~\cite{TMG06}.
In addition, the main story diagram tool, the Fujaba tool suite, has undergone major redesigns the last years; these redesigns also affected the story diagram implementation.
Moreover, new approaches like the Story Diagram Interpreter have emerged.

% Approach
In this technical report, we seek to provide a complete reference of the syntax and semantics of story diagrams.
It consolidates previous publications in a single document.
We provide definitions for the abstract and the concrete syntax as well as the semantics of story diagrams.

%Evaluation
As an example, we show how story diagrams can be used to specify refactoring operations on structural software models like class diagrams.

% Structure
\todojr{Struktur}