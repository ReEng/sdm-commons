\chapter{Concepts} \label{sec:Concepts}

\todoall{Description of elements should follow this structure: What is it for? What does it do? Examples in concrete syntax + explanation}

\section*{Old stuff from rejected paper}
In this section, we will briefly introduce story diagrams and their current features. 
Story diagrams combine
UML activity diagrams and graph transformations by embedding graph replacement rules into the activities.
This allows the activities in Figure \ref{fig:transformationOverview} to be specified formally by graph replacements while preserving the general control flow structure of the example transformation.

In terms of the classification of model transformations proposed by Czarnecki and Helsen~\cite{Czarnecki06}, story diagrams are an endogenous, in-place transformation language.
It has both declarative (pattern matching) and operational elements (specification of control flow):
Its control flow allows a deterministic selection of the graph replacement rules to be applied, with a (non-deterministic) pattern matching in the graph replacement rules.
It can also be used for inter-model transformations to create a new target model from a given source model, as seen in the example given in this paper. 
In order to execute story diagrams, code generation \cite{GBD07} and interpretation \cite{GHS09} are supported.

In the following, we will describe the graph transformations, the so-called story patterns, in Section~\ref{sec:StoryPatterns}.
Afterwards, we will explain how control flow can be modeled by using elements from activity diagrams in Section~\ref{sec:StoryDiagrams}.

	\section{Story Patterns (Christian und Marie)} \label{sec:StoryPatterns}

\todoch{Define concrete syntax for new elements. In particular, check syntax for links.}


\subsection{Story Pattern [CH]}

general idea, purpose, attributes (matching, normal)

\subsection{Objects and Object Variables [MCP]}

\todomcp{each connected component must contain at least one bound object
variable}

%Object variables represent the objects defined in a story pattern.
%They are identified by their name.
%The objects are instances of classes of the underlying classmodel (cp. Section \ref{typedGraphTransformations}).
%Thus, the object variables are typed by classes from this model.

%\begin{figure}[htbp]
%\begin{center}
%  \includegraphics[width=0.25\textwidth]{figures/objectVariable}
%  \caption{An object variable}
%  \label{fig:objectVariable}
%\end{center}
%\end{figure}
%Figure \ref{fig:objectVariable} shows an object variable with the name \texttt{methodDecl} and the type
%\texttt{Method}.

\todomcp{Primitive Variables are part of this section, too; concrete syntax like
object variables; binding expressions for initialization, see figure
\ref{fig:primitiveVariable}; primitive variables are typed over EDataType; they exist beyond the end of the Activity}

\begin{figure}[htbp]
  \centering
  \includegraphics[scale=0.6]{figures/PrimitiveVariable}
  \caption{Primitive variable with value assignment}
  \label{fig:primitiveVariable}
\end{figure}

\todomcp{Links to primitive variables: special LinkVariable, typed over
EStructuralFeature}

\subsection{Links and Link Variables [MCP]}

\subsection{Binding of Variables [MCP]}

%Object variables and link variables have binding operators, binding states, and binding semantics.

\subsubsection{Binding Operators}
%The binding operators define whether an element is to be created, deleted, or just matched.
%After all elements that are defined to be deleted or just matched have been matched, the model ist modified by creating
%and deleting the elements as defined. 

%Elements to be created are marked with the
%label \texttt{CREATE} and elements to be deleted are marked with the label \texttt{DESTROY}.

\subsubsection{Binding States}

\subsubsection{Binding Semantics}

\todomcp{Negative objects, Figure \ref{fig:negativeObjects}:
a) allowed; same semantics for the same situation with link not negative
b) not allowed because graph is not connected
c) allowed
d) allowed because a and b are both bound
e) not allowed
f) allowed; same semantics for the same situation with links not negative
}

\begin{figure}[htbp]
  \centering
  \includegraphics[scale=1]{figures/negativeObjects}
  \caption{Negative Application Conditions}
  \label{fig:negativeObjects}
\end{figure}

\todomcp{Optional objects, Figure \ref{fig:optionalObjects}:
the same as for negative objects?
a) allowed; same semantics for the same situation with link not optional?
b) not allowed because graph is not connected
c) allowed?
d) not allowed
e) allowed
f) allowed; same semantics for the same situation with links not optional
}

\begin{figure}[htbp]
  \centering
  \includegraphics[scale=1]{figures/optionalObjects}
  \caption{Optional Objects}
  \label{fig:optionalObjects}
\end{figure}


\subsubsection{Feasible Binding Combinations}

\todomcp{check Table \ref{tab:bindingCombinations}}
% Table generated by Excel2LaTeX from sheet 'Tabelle1'
\begin{table}[htbp]
  \centering
  \caption{Feasible combinations of binding operators, binding states, and
  binding semantics for object variables}
    \begin{tabular}{|r|r|r|r|}
    \hline
    \textbf{Binding State} & \textbf{Binding Semantics} & \textbf{Binding
    Operator} & \textbf{Result} \\
    \hline
    UNBOUND & MANDATORY & CHECK\_ONLY & yes \\
    UNBOUND & MANDATORY & CREATE & yes \\
    UNBOUND & MANDATORY & DESTROY & yes \\
    UNBOUND & NEGATIVE & CHECK\_ONLY & yes \\
    UNBOUND & NEGATIVE & CREATE & no \\
    UNBOUND & NEGATIVE & DESTROY & no \\
    UNBOUND & OPTIONAL & CHECK\_ONLY & yes \\
    UNBOUND & OPTIONAL & CREATE & yes \\
    UNBOUND & OPTIONAL & DESTROY & yes \\
    \hline
    BOUND & MANDATORY & CHECK\_ONLY & yes \\
    BOUND & MANDATORY & CREATE & no \\
    BOUND & MANDATORY & DESTROY & yes \\
    BOUND & NEGATIVE & CHECK\_ONLY & no \\
    BOUND & NEGATIVE & CREATE & no \\
    BOUND & NEGATIVE & DESTROY & no \\
    BOUND & OPTIONAL & CHECK\_ONLY & no \\
    BOUND & OPTIONAL & CREATE & no \\
    BOUND & OPTIONAL & DESTROY & no \\
    \hline
    MAYBE\_BOUND & MANDATORY & CHECK\_ONLY & yes \\
    MAYBE\_BOUND & MANDATORY & CREATE & no \\
    MAYBE\_BOUND & MANDATORY & DESTROY & yes \\
    MAYBE\_BOUND & NEGATIVE & CHECK\_ONLY & no \\
    MAYBE\_BOUND & NEGATIVE & CREATE & no \\
    MAYBE\_BOUND & NEGATIVE & DESTROY & no \\
    MAYBE\_BOUND & OPTIONAL & CHECK\_ONLY & no \\
    MAYBE\_BOUND & OPTIONAL & CREATE & no \\
    MAYBE\_BOUND & OPTIONAL & DESTROY & no \\
    \hline
    \end{tabular}%
  \label{tab:bindingCombinations}%
\end{table}%

\todomcp{see albert's diss for example for optional-create}
\todomcp{table for link variables and for object set variables}

\subsection{Using Object Attributes [CH]}

object constraints, attribute assignments

\subsection{Object Sets [MCP]}

explain objectSetVariables, set size expressions

\todomcp{object sets and binding operators/states/semantics}

\todomcp{If we bind an object set, can we use the bound object in other story pattern? E.g. to insert all elements bound by the object set into a container via a containment link? (See Figure \ref{fig:reuseObjSet}).}
\tododt{Yes, but I would use another concrete syntax (see Figures~\ref{fig:reuseObjSet1}, \ref{fig:reuseObjSet2}, and \ref{fig:reuseObjSet1}).}

\begin{figure}[p]
	\begin{minipage}{.45\textwidth}
		\centering
		\includegraphics[scale=.8]{figures/ReuseObjectSet}
  	\caption{Reusing Object Sets}
  	\label{fig:reuseObjSet}
	\end{minipage}
  \hfill
  \begin{minipage}{.45\textwidth}
  	\centering
		\includegraphics[scale=.8]{figures/ReuseObjectSet1}
  	\caption{Reuse objects in a set}
  	\label{fig:reuseObjSet1}
	\end{minipage}
\end{figure}

\begin{figure}[p]
	\begin{minipage}{.45\textwidth}
		\centering
		\includegraphics[scale=.8]{figures/ReuseObjectSet2}
  	\caption{Add an object to a set}
  	\label{fig:reuseObjSet2}
	\end{minipage}
  \hfill
  \begin{minipage}{.45\textwidth}
  	\centering
		\includegraphics[scale=.8]{figures/ReuseObjectSet3}
  	\caption{Add all objects from a set to a container}
  	\label{fig:reuseObjSet3}
	\end{minipage}
\end{figure}

\todomcp{An object set contains no ObjectSetSizeExpression and no object is matched into the object set: ObjectSet is interpreted as optional and the matching succeeds.}

\todomcp{All operators for comparison are allowed: <, <=, >, >=, = !=}


\begin{figure}[p]
	\begin{minipage}{.45\textwidth}
		\centering
		\includegraphics[scale=.8]{figures/ObjectSetSize}
  	\caption{Object Set Size}
  	\label{fig:objSetSize}
	\end{minipage}
  \hfill
  \begin{minipage}{.45\textwidth}
  	\centering
		\includegraphics[scale=.8]{figures/IsomorphismInObjectSets}
  	\caption{Isomorphism in Object Sets}
  	\label{fig:isoObjSet}
	\end{minipage}
\end{figure}

\todomcp{What happens if one pattern contains two object sets that may possibly contain the same objects. Consider the pattern in Figure \ref{fig:isoObjSet}. If c1 and c2 share the same super classes, su1 and su2 contain the same objects. Is this allowed? Isomorphic matching would normally forbid this.}

\tododt{I would allow this which would comply with our isomorphic matching.
But in this case it is non-deterministic how the objects are matched to the set nodes (assuming the classes are already bound).
A maybe constraint could allow to match the same objects to both sets.}

\subsection{Special Link Kinds [CH]}

\subsubsection{Containment Relations [CH]}

\todoch{What is the semantics of a containment link? Current understanding: an element is contained in a container. What is the difference to to-many references which are containments?}
\tododt{Containment links do not correspond to any association or reference. They only describe containments of objects in containers or set nodes.}

\todoch{Which classifiers are allowed for ContainerVariables? This should not only be the EMF collection types EList and EMap. Does a ContainmentLink need a reference as a type like normal LinkVariables?}
\tododt{The contaiment links do not need any reference. I would suggest to allow any subtype of java.util.Collection (and java.util.List in case that link order constraints are used), but not java.util.Map! This is something different and needs a key for containment. Maybe we need special containment links for maps that include a key.}

\subsubsection{Paths [MCP]}

\subsubsection{Link Constraints [CH]}

\todoch{What is the concrete Syntax for this?}

\tododt{I have a suggestion in Figure~\ref{fig:linkConstraints}.}

\begin{figure}[p]
  \centering
  \includegraphics[scale=.8]{figures/LinkConstraints1}
  \caption{Link order constraints FIRST and DIRECT\_SUCCESSOR}
  \label{fig:linkConstraints}
\end{figure}

Link constraints are only applicable to link variables that reference an ordered to-many reference. 
\begin{itemize}
  \item FIRST = matches the first element in the list, requires one link variable
  \item LAST = matches the last element in the list, requires one link variable
  \item INDEX = matches the element at the specified index, requires one link variable
  \item DIRECT\_SUCCESSOR = requires two link variables, target of the second one must be located directly after the target of the first one in the list
  \item INDIRECT\_SUCCESSOR = requires two link variables, target of the second one must be located somewhere after the target of the first one in the list
\end{itemize}

\todoch{Is this semantic description correct/complete?}

\subsubsection{Maybe Links}

Disables the isomorphism check for two object variables, these two object variables may be matched to the same object.

\todoch{What is the concrete Syntax? Using a special pattern constraint as proposed in Alberts Habil is very low-level. Alternative version is proposed in Figure \ref{fig:maybeLink}.}

\begin{figure}[htbp]
  \centering
  \includegraphics[scale=.8]{figures/MaybeLink}
  \caption{Maybe Link}
  \label{fig:maybeLink}
\end{figure}

\subsection{Pattern Constraints [CH]}

\todoch{What happens when a pattern constraint is placed inside a for each story pattern (not inside a node in that pattern)? Proposal: The particular match must fulfill the pattern constraint, if it does not fulfill the pattern constraint, the match is rejected and the iteration continues.}
\tododt{Exactly. In this case, it is a kind of post condition that has to be satisfied at the end of the matching step.}

\subsection{Pattern Fragments [MCP]}

patterns contained in other patterns, negative, semantics? review enhanced story patterns from Diss Florian Klein

\todomcp{Should contained pattern be marked as forEach? Idea for semantics: first the part of the pattern outside the forEach pattern is matched, then the forEach subpattern is applied to any match that may be located, the variables bound in a forEach subpattern may not be used in subsequent activities}

\tododt{This is somewhat confusing.
As I understand them, subpatterns are ordinary story patterns within another story pattern.
They are surrounded by a fragment box and can be labeled with a name (see Figure~\ref{fig:labeledSubPattern}).
Special types of such subpatterns are negative application condition fragments (NACs), set fragments, and optional fragments.
As far as I know, we did not plan to add $\forall$ and $\exists$ fragments, did we?
These are only used in SDDs and TSSDs which are constraint languages.
}

\begin{figure}[htbp]
  \centering
  \includegraphics[scale=1.0]{figures/ContainedPattern}
  \caption{Different Kinds of Contained Patterns}
  \label{fig:containedPattern}
\end{figure}

\todomcp{Should contained pattern be marked as optional? Is currently possible in the meta-model. Idea for semantics: Whole pattern must be found, if found, variables may be used in subsequent activities, if pattern may not be found as a whole, matching still succeeds but all variables in the subpattern are not bound in subsequent activities.}
\tododt{I would say, contained patterns are mandatory in general (or are NAC/optional/set in case of the according fragment).}

\begin{figure}[htbp]
  \centering
  \includegraphics[scale=1.0]{figures/SubPatterns2}
  \caption{Labeled sub pattern}
  \label{fig:labeledSubPattern}
\end{figure}

\begin{figure}[htbp]
  \centering
  \includegraphics[scale=1.0]{figures/SubPatterns1}
  \caption{Hierarchies of NAC, set, and optional sub patterns}
  \label{fig:subPatternHierarchies}
\end{figure}

\todomcp{How deep may patterns be nested? What is the semantics of alternating binding semantics of sub-patterns, e.g. negative in optional in negative and so on.}
\tododt{I would prefer to allow arbitrarily deep nestings and would suggest to interpret the fragments in the order from outside to inside. Example (see Figure~\ref{fig:subPatternHierarchies}): You match a super class \emph{superClass} of \emph{myClass} and ensure that \emph{superClass} has no attribute. Then you you match all methods \emph{new} (outer set fragment) that have no class as their type (enclosed NAC fragment). Now you match for each of these methods all parameters (enclosed set fragment) that have \emph{myClass} as their type. Furthermore, you try to find a path from the matched \emph{new} method to a return statement (optional fragment).}


\subsection*{Old stuff from rejected paper}

Story patterns describe graph replacement rules that can be embedded into the activities of a story diagram. They are based on labeled, attributed graphs that are extended by a type model \cite{FNTZ00}. 
The types and references that are specified in the type model are used to type the nodes and edges within the story pattern.
Type models for story diagrams can be created, e.g., by using EMF Ecore \cite{SBP+08}.
In our example, we will use the meta-models shown in Figures~\ref{fig:sourceMetamodel} and~\ref{fig:targetMetamodel} as type models. The type model supports inheritance and polymorphism, i.e., a node of type \fe{Classifier} matches objects of types \fe{Classifier}, \fe{Class}, and \fe{PrimitiveType}.
This allows specifying graph replacement rules for object-oriented models.

In order to provide a concise notation, story patterns apply a short-hand notation depicting left-hand side and right-hand side in one graph. Nodes and edges being created (or deleted) are annotated with \small \verb|<<create>>| \normalsize (or  {\small \verb|<<destroy>>|\normalsize}, respectively). The matching of story patterns in a host graph requires an isomorphic matching of the pattern's left-hand side in the host graph, i.e., two nodes of the pattern may not be matched to the same node in the host graph \cite{FNTZ00,Roz97}. The matching is performed with respect to the types of the type model. The deletion of nodes is applied according to the Single Pushout Approach (SPO, \cite{Roz97}), i.e., dangling edges resulting from the deletion of nodes are deleted as well.

Figure~\ref{fig:SP} shows an example of a story pattern that simply adds a class to the set of classifiers of the source system (cf. Figure~\ref{fig:sourceMetamodel}).

\begin{figure}[htbp]
\begin{center}
  \includegraphics[width=0.25\textwidth]{figures/StoryPattern}
  \caption{Simple story pattern}
  \label{fig:SP}
\end{center}
\end{figure} 
	
	\section{Story Diagrams} \label{sec:StoryDiagrams}

\subsection*{Old stuff from rejected paper}
Story diagrams are an extension of UML 1.4 activity diagrams \cite{UML} that embed story patterns into the activities.
That allows to model basic control flow structures like branches or loops.
Figure \ref{fig:controlFlow} shows a story diagram that embeds the story pattern of Figure \ref{fig:SP} into one of its activities.
The purpose of the story diagram is to create a new class in the source system if no class with the name given by the parameter \fe{className} already exists. 

\begin{figure}[tbp]
\begin{center}
  \includegraphics[width=0.7\textwidth]{figures/ControlFlow}
  \caption{Control flow in story diagrams}
  \label{fig:controlFlow}
\end{center}
\end{figure}

In the first activity, the embedded story pattern tries to bind a class with the respective name in the source system. The \fe{sourceSystem} is given as a parameter to the story diagram and can be used as a \emph{bound} node by referring to the name of the parameter. A bound node is signified in the concrete syntax by omitting its type information. Then, the story pattern tries to bind an object of type \fe{Class} to the \emph{unbound} node named \fe{c} such that the attribute condition is fulfilled. 
If this pattern can be matched successfully, i.e., the class already exists, the activity is left via the \fe{[success]} transition and the story diagram terminates.
If no such class can be found, the matching fails and the activity is left via the \fe{[failure]} transition.
Then, the second activity creates the class, links it to the source system, and sets the respective name. 
Additionally, it is possible to add boolean conditions and an \fe{[else]} to the transitions to model more specific conditions.

In general, an initial matching is established by the parameters. This matching is extended by the story patterns in the activities. Then, the matching is propagated to the next activity along the transitions. If a story pattern fails, the current matching is not changed. In subsequent activities, an object previously bound to a node \emph{c} can be referenced using a bound node with name \emph{c}.

The specification of the transformation outlined in Figure~\ref{fig:transformationOverview} can only be accomplished by specifying loops because it requires iterating over all classifiers of the system and all methods of the classes. Loops can be modeled using \emph{forEach activities}.
The story patterns in forEach activities are matched as long as new matchings can be found.
They are visualized by a double border line as depicted in Figure~\ref{fig:forEach}. The transformation formalizes the informal description of Figure~\ref{fig:transformationOverview} using the current features of story diagrams.

\begin{figure}[htb]
\begin{center}
  \includegraphics[width=\textwidth]{figures/ForEach}
  \caption{Example of a complex transformation including several forEach activities}
  \label{fig:forEach}
\end{center}
\end{figure}

The story diagram creates a target system in its first activity. Then, all classes of the source system \fe{sourceSystem} are matched. For each match that has been found, the forEach activity is left via the \fe{[each time]} transition. Then, the third activity creates a respective class and its header in the target system. Afterwards, all methods of the class \fe{c} of the source system are matched. Again, the forEach activity is left for each new match using the \fe{[each time]} transition. After all methods have been transformed, the control flow returns to the activity \fe{Bind classes} to bind the next class. It is required that a control flow that has left a forEach activity eventually returns to this activity to obtain a correct story diagram.

After transforming the classes, all primitive types of the source model have to be transformed. We omit the details here due to space limitations. 
Finally, the target system object \fe{targetSystem} is returned by the story diagram as indicated by the annotation \fe{return targetSystem} on the stop node.

Hitherto, story diagrams only support proprietary calls to library functions called collaboration statements. In Figure \ref{fig:forEach}, the activity \fe{InvokeCodeGen} contains such a call to the code generator. The call only is a string expression that does neither allow type checking of the input parameters nor getting a matching of return values, i.e., the return values cannot be used in the story diagram.

\subsection{Applications of Stroy Diagrams (Dietrich)}
2 worlds: with and without methods\ldots

\subsection{General composition of Story Diagrams}
\begin{itemize}
  \item Based on UML 1.5 activity diagrams
  \item Grammar for story diagrams (based on grammar in diploma thesis of Thomas Klein (1999))
\end{itemize}

\subsection{Calls}

\todomvd{Wie sehen Calls (Method/Activity) aus?}

	\section{Expressions} \label{sec:Expressions}
	
	\input{Concept_Templates.tex}
	
 