\section{Story Patterns (Christian und Marie)} \label{sec:StoryPatterns}

\todoch{Define concrete syntax for new elements. In particular, check syntax for links.}
\todoch{What happens when a constraint is placed inside a for each story pattern (not inside a node in that pattern)?}

\subsection{Story Pattern [CH]}

general idea, purpose, attributes (matching, normal)

\subsection{Objects and Object Variables [MCP]}

object variables

\subsection{Links and Link Variables [MCP]}

\subsection{Binding of Variables [MCP]}

Explain enums BindingOperator, BindingState, and BindingSemantics

\subsection{Using Object Attributes [CH]}

object constraints, attribute assignments

\subsection{Object Sets [MCP]}

explain objectSetVariables, set size expressions

\subsection{Special Link Kinds [CH]}

\subsubsection{Containment Relations [CH]}

\subsubsection{Paths [MCP]}

\subsubsection{Link Constraints [CH]}

\subsection{Pattern Constraints [CH]}

\subsection{Contained Pattern [MCP]}

patterns contained in other patterns, negative, semantics? review enhanced story patterns from Diss Florian Klein

\subsection*{Old stuff from rejected paper}

Story patterns describe graph replacement rules that can be embedded into the activities of a story diagram. They are based on labeled, attributed graphs that are extended by a type model \cite{FNTZ00}. 
The types and references that are specified in the type model are used to type the nodes and edges within the story pattern.
Type models for story diagrams can be created, e.g., by using EMF Ecore \cite{SBP+08}.
In our example, we will use the meta-models shown in Figures~\ref{fig:sourceMetamodel} and~\ref{fig:targetMetamodel} as type models. The type model supports inheritance and polymorphism, i.e., a node of type \fe{Classifier} matches objects of types \fe{Classifier}, \fe{Class}, and \fe{PrimitiveType}.
This allows specifying graph replacement rules for object-oriented models.

In order to provide a concise notation, story patterns apply a short-hand notation depicting left-hand side and right-hand side in one graph. Nodes and edges being created (or deleted) are annotated with \small \verb|<<create>>| \normalsize (or  {\small \verb|<<destroy>>|\normalsize}, respectively). The matching of story patterns in a host graph requires an isomorphic matching of the pattern's left-hand side in the host graph, i.e., two nodes of the pattern may not be matched to the same node in the host graph \cite{FNTZ00,Roz97}. The matching is performed with respect to the types of the type model. The deletion of nodes is applied according to the Single Pushout Approach (SPO, \cite{Roz97}), i.e., dangling edges resulting from the deletion of nodes are deleted as well.

Figure~\ref{fig:SP} shows an example of a story pattern that simply adds a class to the set of classifiers of the source system (cf. Figure~\ref{fig:sourceMetamodel}).

\begin{figure}[htbp]
\begin{center}
  \includegraphics[width=0.25\textwidth]{figures/StoryPattern}
  \caption{Simple story pattern}
  \label{fig:SP}
\end{center}
\end{figure}