\chapter{Foundations}
\label{sec:foundations}

\todoall{What else do we need to address in this chapter?}

\section{Graphs and Graph Transformations}
\label{sec:foundations:simpleGTS}

Graphs consist of nodes and edges where an edge always connects two nodes. Nodes are used to represent objects and edges denote relationships between these objects. In course of this document, we will assume edges to be directed, i.e., they have a source node and a target node. In the most simple case, neither nodes nor edges have a predefined semantics \cite{Roz97}.

\begin{figure}[htbp]
  \centering
  \includegraphics[scale=1.5]{figures/SimpleGraph}
  \caption{Simple Graph}
  \label{fig:simpleGraph}
\end{figure}

Figure \ref{fig:simpleGraph} shows an example of a simple graph with three nodes and four edges. The nodes are visualized a circles, the edges are visualized as arrows. An edge may have the same node as a source and target node. Such an edge is called a \emph{self-edge}.

\emph{Graph transformation rules} specify allowed modifications of graphs. They consist of a left-hand side (LHS), a right-hand side (RHS), and a so-called rule morphism. Both, the LHS and the RHS are graphs while the rule morphism specifies which nodes of the LHS and RHS are considered to be the same. This information is required for the application of a graph transformation rule to a graph.

\begin{figure}[htbp]
  \centering
  \includegraphics[scale=1.5]{figures/SimpleGTRule}
  \caption{Simple Graph Transformation Rule}
  \label{fig:simpleGTRule}
\end{figure}

Figure \ref{fig:simpleGTRule} shows an example of a graph transformation rule. The LHS contains only node node with a self-edge. The RHS contains two nodes connected by an edge where the right node of the RHS has a self-edge as well. The rule morphism is visualized by the grey, dotted arrow. It specifies that the node of the LHS and the left node of the RHS are considered to be the same.

The application of a graph transformation rule to a graph is called a \emph{graph transformation} \cite{EEPT06}. The graph on which the rule is to be applied is called the \emph{host graph}.
The application of a graph transformation rule to a graph is performed in three steps. In the first step, an occurrence of the LHS of the graph transformation rule in the host graph is searched. Such an occurrence is called a \emph{match} of the graph transformation rule. If a match has been found, all nodes and edges that occur in the LHS but not in the RHS are deleted from the host graph. In this step, the rule morphism is used to decide which nodes do not occur in the RHS. In the third step, all nodes and edges that occur in the RHS but not in the LHS are added to the host graph. After the application of the graph transformation rule, there exists a match of the RHS into the host graph.

\begin{figure}[htbp]
  \centering
  \includegraphics[width=\linewidth]{figures/GTApplication}
  \caption{Application of a Graph Transformation Rule}
  \label{fig:GTApplication}
\end{figure}

Figure \ref{fig:GTApplication} shows an example of a graph transformation that applies the graph transformation rule of Figure \ref{fig:simpleGTRule} to the graph of Figure \ref{fig:simpleGraph}. The matching of the LHS into the host graph is visualized by a gray, dotted line. Then, the graph transformation rule deletes the self-edge from this node. Afterwards, a new node with a self-edge is created and connected to the previously matched node by an edge. The match of the RHS into the host graph after the rule application is again shown by gray, dotted lines.

In the field of algebraic graph transformations, the two most popular approaches for applying a graph transformation rule to a graph are the
\emph{double-pushout approach} \cite{Roz97} and the \emph{single-pushout
approach} \cite{Roz97}. The definition of story diagrams follows the
single-pushout approach. Besides the more theoretical differences the two
approaches differ in the handling of two special situations that might occur
upon rule application.

The first situation is the following. Assume the left-hand side of a rule
consists of two nodes. The first node is to be deleted and the second one is
to be preserved. Both of these nodes may be matched to the same node in the host
graph. In this situation, it is not clear if the node in the host graph is to be
deleted or preserved. The double-pushout approach explicitly forbids the application of the rule in such
situations. The single-pushout approach allows such situations and gives
deletion priority over preservation.

The second situation deals with dangling edges. It occurs if a certain node is
to be deleted but some of its incident edges are to be preserved. The
transformation would lead to a non-valid graph in which the edges would not have
either a source or a target node. The double pushout approach does not allow
such situations and instead requires that incident edges are explicitly
deleted. The single-pushout approach allows such situations and implicitly
deletes edges if one of the source or target nodes are deleted.

In general, matches of graph transformation rules are homomorphisms of the LHS of the rule to the host graph. That allows to match two nodes of the LHS to the same node of the host graph leading to first situation mentioned above. Such situations may be prevented by using isomorphisms for matching the LHS. Then, each node of the LHS must be matched to a unique node of the host graph. Thus, using isomorphic matchings solves the first situation when using single pushouts.

\section{Typed Attributed Graph Transformations}
\label{sec:foundations:typedAttrGTS}

Graphs and according graph transformations as introduced in Section \ref{sec:foundations:simpleGTS} are a very basic approach to modeling behavior. When using graph transformations for modeling behavior for object-oriented software or as a foundation for defining the semantics of modeling languages, it is necessary to distinguish different types of nodes and edges in a graph in order to give them semantics. 

Therefore, story diagrams are based on typed attributed graph transformations \cite{EEPT06}. Typed attributed graph transformations introduce a type graph and node attributes. The type graph defines different types of nodes and edges and it defines which types of edges are allowed for which type of nodes. Additionally, nodes may carry attributes like, e.g., objects in an object-oriented programming language. Accordingly, the type graph specifies inheritance relations between types of objects which are also known from object oriented programming languages.

If the graph transformation transforms a model based on a given type graph into a model based on the same type graph, the transformation is called \emph{endogenous}. Otherwise, i.e., the transformation transforms a model based on a given type graph into a model based on another type graph, the transformation is called \emph{exogenous.} Story diagrams are endogenous graph transformations.

\section{Type Graph used in this Report}
\label{sec:typeGraph}

\begin{figure}[htbp]
  \centering
  \includegraphics[width=\linewidth]{figures/gast-mm}
  \caption{Type graph of a generalized abstract syntax tree (GAST).}
  \label{fig:gast-mm}
\end{figure}

The type graph used in the examples in this report describes the structure of an abstract syntax tree. In particular it is an updated and slightly simplified version of the generalized abstract syntax tree (GAST) meta model developed in the QBench project \cite{QBench}. The GAST was developed to provide a unified syntax tree model for different programming languages like Java, C, and C++.
Figure~\ref{fig:gast-mm} shows an excerpt of that meta model. Especially some specialized sub classes have been omitted for clarity reasons.

\begin{description}
\item[Root] The \fe{Root} element is the central element of every GAST model. All other elements are reachable from the \fe{Root} node via composition relations.
\item[File] Elements of the GAST, e.g., classes and packages, can be assigned to files in the file system. A \fe{File} element holds references to those classes and packages and a String containing the path to the file.
\item[Package] Similar to packages in Java, the \fe{Package} element provides name spaces and visibilities. A \fe{Package} element can contain other packages, classes, global variables, and functions.
\item[GASTType] The \fe{GASTType} element represents data types like primitive data types and classes. The attribute \fe{qualifiedName} contains the unique, fully qualified name of the type.
\item[GASTClass] Classes are represented by the element \fe{GASTClass} in the GAST and are a sub type of the \fe{GASTType}. A \fe{GASTClass} holds references to its methods, attributes, and inner classes. A \fe{GASTClass} can be assigned to a \fe{Package}.
\item[Function] \fe{Function} is the super type for all executable operations. In addition to a name attribute, a \fe{Function} can have a number of local variables and formal parameters. The return type of a \fe{Function} is determined by its \fe{DeclarationTypeAccess}, a sub class of \fe{Access} (not shown in Figure~\ref{fig:gast-mm}). A \fe{Function} always contains a block statement which, in turn, can contain other statements.
\item[GlobalFunction] A \fe{GlobalFunction} element represents a globally accessible operation, i.e., an operation that does not belong to a class. They can be assigned to a name space defined by a package. For example, C functions are represented by \fe{GlobalFunctions}.
\item[Method] Functions that belong to an class are represented by \fe{Method} elements, a sub type of \fe{Function}.
\item[Variable] \fe{Variable} is a super type for all kinds of variables. A \fe{Variable} always has a name and a type.
\item[LocalVariable] \fe{LocalVariables} are variables that are contained in a \fe{Function}.
\item[FormalParameter] \fe{FormalParameters} are variables that represent the parameters of a \fe{Function},
\item[GlobalVariable] \fe{GlobalVariables} are variables that are globally accessible within a given scope. The scope is determined by the package in which the \fe{GlobalVariable} is contained.
\item[Field] The \fe{Field} element represents class variables. Therefore it is contained in a \fe{GASTClass}.
\item[Statement] A \fe{Function} consists of a number of \fe{Statements}. There are multiple sub classes of \fe{Statement} which represent the different kinds of statements. Most of them are omitted here. A \fe{Statement} can contain a number of \fe{Accesses}.
\item[BlockStatement] The \fe{BlockStatement} is a special kind of statement which can contain other \fe{Statements}. It is the root element of all \fe{Statements} contained within a \fe{Function}.
\item[Access] An \fe{Access} represents the use of a \fe{Variable} or a \fe{Function}. It always belongs to a certain \fe{Statement}.
\item[FunctionAccess] A \fe{FunctionAccess} represents the use of a \fe{Function} in a \fe{Statement} and therefore references the accessed \fe{Funtion} element.
\item[VariableAccess] A \fe{VariableAccess} represents the use of a \fe{Variable} in a \fe{Statement} and therefore references the accessed \fe{Variable} element.
\end{description}
