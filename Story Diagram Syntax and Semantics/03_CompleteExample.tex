\chapter{Complete Example}

To illustrate our concept, we use a language transformation scenario: Suppose we have a model representing the abstract syntax graph (ASG) of a program. The ASG is language-specific, of course. If we want to generate code from the ASG for a target language other than the language on which the given ASG is based, one option is to transform the ASG first.
One example of this would be a transformation from Java to C++.

\begin{figure}[ht]

\begin{minipage}[b]{0.42\linewidth}
	\centering
	\includegraphics[scale=0.7]{figures/sourceMetamodel}
\caption{Source meta-model}
\label{fig:sourceMetamodel}
\end{minipage}
\hfill
\begin{minipage}[b]{0.55\linewidth}
\centering
	\includegraphics[scale=0.7]{figures/targetMetamodel}
\caption{Target meta-model}
\label{fig:targetMetamodel}
\end{minipage}

\end{figure}


Figure~\ref{fig:sourceMetamodel} shows a simplified meta-model for an ASG which acts as the source meta-model for our exemplary transformation scenario. A \fe{SourceSystem} consists of a number of \fe{Classifiers} which can either be \fe{PrimitiveTypes} or \fe{Classes}. Each \fe{Class} can have a number of \fe{Methods}.

The target meta-model for the transformation, shown in Figure~\ref{fig:targetMetamodel}, is slightly more complex.
The \fe{TargetSystem} consists of a number of \fe{CClassifiers} which are either \fe{DataTypes} or \fe{CClasses}.
An \fe{CClass} can contain a number of \fe{Functions}. In addition, each \fe{CClass} has a corresponding \fe{Header} which contains a \fe{FunctionDeclaration} for each \fe{Function} of the \fe{CClass}.


\begin{figure}[htbp]
\begin{center}
  \includegraphics[width=\textwidth]{figures/transformationOverview}
  \caption{An activity diagram describing the example transformation}
  \label{fig:transformationOverview}
\end{center}
\end{figure}

Figure~\ref{fig:transformationOverview} shows an activity diagram which gives an overview of the transformation from a source ASG to a target ASG.
At first, the target system has to be created. Each classifier in the source system has to be transformed into a corresponding classifier in the target system. While primitive types can be transformed easily, headers have to be created for all transformed classes. For each method, a function has to be created and a corresponding function declaration has to be added to the correct header. In the end, the target system can be passed to the code generation mechanism.

\subsection{Requirement}

Das Beispiel muss enthalten:
\begin{itemize}
  \item Komplexer Kontrollfluss (success und failure, forEach, exception und finally, boolean conditions), verschiedene Node Types (normal, forEach, start, stop, Junction)
  \item StatementNode und TextualExpression
  \item Negative Anwendungsbedingung und komplexe negative Anwendungsbedingung (Subpattern eines Story Pattern)
  \item Calls inkl. polymorphic dispatching
  \item Expressions (Attribute Assigments, Attributvergleich beim Matching, Constraint eines Story Pattern), LinkConstraints (z.B. neue Methode als letzte in die Liste einsortieren, neue Methode in alphabetischer Reihenfolge einsortieren)
  \item SetObjects und Constraints fuer diese
  \item Structured Nodes und Flow Stops
  \item Pattern Matching: Optional, Maybe bound, Optional create, Optional destroy?
  \item Maybe construct, um Isomorphietest fuer zwei gebundene Objekte zu deaktivieren
  \item Templates
\end{itemize}