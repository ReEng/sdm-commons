\chapter{Foundations}

\todoall{What else do we need to address in this chapter?}

\section{What are Story Diagrams?}

Typed, attributed graph transformations with inheritance\ldots

\subsection{Single Pushout (Christian und Julian)}

In the field of graph transformations the two most popular approaches are the
\emph{double-pushout approach} \cite{Roz97} and the \emph{single-pushout
approach} \cite{Roz97}. The definition of story diagrams follows the
single-pushout approach. Besides the more theoretical differences the two
approaches differ in the handling of two special situations that might occur
upon rule application.

The first situation is the following. Assume the left-hand side of a rule
consists of two nodes. The first node is to be deleted and the second one is
to be preserved. Both of these nodes may be matched to the same node in the host
graph. In this situation it is not clear if the node in the host graph is to be
deleted or preserved. The double-pushout approach explicitely forbits such
situations. The single-pushout approach allows such situations and gives
deletion priority over preservation.

The second situation deals with dangling edges. It occurs if a certain node is
to be deleted but some of its incident edges are to be preserved. The
transformation would lead to a non-valid graph in which the edges would not have
either a source or a target node. The double pushout approach does not allow
such situations and instead requires that incident edges are explicitely
deleted. The single-pushout approach allows such situations and implicitely
deletes edges if one of the source or target nodes are deleted.
